\section{Motivating Example}\label{sec:mot}
A goods brokerage company \cite{PetermannJMR14} trades items between producers (vendors) and retailers (customers): each activity starts with a vendor sending a sales quotation to a customer. Such quotation considers both the item's purchase price as well as the additional revenues. The process stops whether the customer rejects the offer. Otherwise, the order is confirmed and the item is scheduled for delivery and, when ready, is sent to a logistic operator. In this occasion, the sales invoice and the sales order is sent to the retailer. Next, both the producers and the retailers might rank the items on a scale from 0 to 10, where 0 denotes a despicable product while 10 denotes an excellent product. A retailer ranking a product extremely low allows it to file a complaint ticket to the brokerage company which might grant a refund. 

In such scenario, deviant traces are traces that either do not respect the company's rules, or traces that will potentially lead to retailers' complaints. With respect to the first one, a company must send a product only after receiving the offer's acceptance. This constraint can be easily modelled with a Declare constraint $\Sdeclare{Precedence}{AcceptOffer}{SendToLogistics}$, as such constraint allow to express temporal constraints within trace activities. With respect to the second one, we can observe that a retailer might file a complaint ticket to the brokerage company if he ranks the received product as of bad quality. Next, if the producer assesses that also the product is bad, then a refund is triggered. E.g., the first part of the complaint handling can be modelled as $\Pfdeclare{Precedence}{RP}{ComplaintTicket}{\texttt{RP}.\textup{quality}\leq 3}$, where \texttt{RP} stands for ``Receive Product''. Please observe that this constraints requires to enrich the Declare templates with data-aware predicates, such as  $\texttt{RP}.\textup{quality}\leq 3$ for quantifying the product quality. 


Last, we can observe that some temporal information cannot be expressed by data-agnostic Declare templates. For example, a late delivery complaints occur if the date of a product is greater than the agreed time to deliver it in the previous sales order. This situation cannot be directly expressed as a $\textsf{Precedence}$, as we also need to test the timestamps as both data and event timestamps. Albeit this task requires to represent temporal information within the data perspective \cite{MultiPerspective}, , this would require to express \textit{join} conditions within the single Declare template of interest. In the present work, we discard the possibility of expressing such join constraints: please observe that this is a quite common consideration within the spectrum of Business Process Management, and therefore we will continue to work under this working assumption \cite{10.1007/978-3-642-40176-3_8}. Nevertheless, we are planning to extend the proposed approach so to enable such intra-Declare template join condition.