\section{Related Work}
\label{sec:related}

Most of the conformance checking techniques reported in the scientific literature are based on procedural models. State of the art examples of these approaches are presented in \cite{DBLP:journals/tosem/CookW99,DBLP:journals/is/RozinatA08,DBLP:conf/edoc/AdriansyahDA11,DeLeoni2013,BPM-14-07,DBLP:conf/apn/Aalst12,DBLP:journals/dpd/Aalst13,DBLP:conf/otm/LeoniMCA14,DBLP:journals/is/Munoz-GamaCA14,Knuplesch2010,Awad2009:Compliance,RamezaniTaghiabadi2013,Taghiabadi2014}.

In \cite{DBLP:journals/tosem/CookW99,DBLP:journals/is/RozinatA08}, for the first time, the concept of conformance checking with respect to (procedural) process models was investigated. In \cite{DBLP:conf/edoc/AdriansyahDA11}, the authors introduce conformance checking augmented with the notion of alignments.
Alignment-based approaches have also been used in \cite{DeLeoni2013,BPM-14-07,RamezaniTaghiabadi2013,Taghiabadi2014}, where the authors propose techniques for conformance checking with respect to time- and data-aware procedural models.
Since the latter approach first processes the control flow and then tests data conditions, this approach cannot be easily applied to the case of data-aware Declare. Indeed, in case the reference model contains constraints that are in conflict when considering the control flow in isolation (for example constraints requiring the existence and the absence of the same activity $\texttt{payment}$) this approach would stop after the first step because it cannot find any control flow-based alignments. However if the two constraint are not in conflict when considering the data perspective an alignment should be found (for example activity $\texttt{payment}$ is forbidden with a negative amount but must occur with a positive amount).


In recent years, an increasing number of researchers are focusing on the conformance checking with respect to declarative models.
For example, in \cite{Chesani2009}, an approach for compliance checking with respect to \emph{reactive business rules} is proposed. Rules, expressed using Declare, are mapped to Abductive Logic Programming, and Prolog is used to perform the validation. The approach has been extended in \cite{Montali2010:Choreographies}, by mapping constraints to LTL, and evaluating them using automata. The entire work has been contextualized into the service choreography scenario. In \cite{Grando2012,Grando2013}, Grando et al.\ used Declare to model medical guidelines and to provide semantic (i.e., ontology-based) conformance checking measures.
The work described in \cite{Leoni2012,DeLeoni2014} consists in converting a Declare model into an automaton and perform conformance checking of a log with respect to the generated automaton. The conformance checking approach is based on the concept of alignment and as a result of the analysis each trace is converted into the most similar trace that the model accepts. All the above approaches are data agnostic


More recently, in \cite{Borrego2014}, the authors present an approach where the data perspective for conformance checking with Declare is expressed in terms of conditions on global variables disconnected from the specific Declare constraints expressing the control flow. In \cite{BurattinMS16}, the authors provide an algorithmic framework to efficiently check the conformance of MP-Declare with respect to event logs. These approaches do not provide repair strategies. 