\section{Related Work}
\label{sec:related}

Most of the conformance checking techniques reported in the scientific literature are based on procedural models. In \cite{DBLP:conf/edoc/AdriansyahDA11}, for the first time, the authors introduce conformance checking augmented with the notion of alignments. %The work described in \cite{LeoniMA12} presents a (data agnostic) conformance checking approach based on the concept of alignment for declarative models.

\added{The majority of  data-aware conformance checking dealt with procedural models; in this branch, \cite{LeoniA13} combined the A* algorithm for alignment-based control-flow conformance checking in \cite{LeoniMA12} with Integer Linear Programming for data conformance checking. Similarly, data perspective for con\-form\-ance checking with Declare could be expressed in terms of conditions on global process variables disconnected from the specific Declare constraints expressing the control flow \cite{Borrego014}.  As pointed out in \cite{MultiPerspective}, such solution provides misleading results, as often control-flow and data prospective are closely inter-related in real-world scenarios models \cite{PetermannJMR14}. As a result, a cost function considering both data and control-flow discrepancies is required. Still, it is widely known \cite{AdriansyahDA10} that some trace-alignment strategies do not provide correctness guarantees, as perfectly fitting traces might be assessed as non-fitting executions.}


%Alignment-based approaches have also been used in \cite{MultiPerspective}, where the authors propose techniques for conformance checking with respect to data-aware procedural models. Since the latter approach first processes the control flow and then tests data conditions, it cannot be easily applied to the case of data-aware Declare. Indeed, in case the reference model contains constraints that are in conflict when considering the control flow in isolation (e.g., constraints requiring the existence and the absence of the same activity $\texttt{payment}$) this approach would stop after the first step because it cannot find any control flow-based alignment given the inconsistency of the model. \added{It has been also proved that such a solution does not correctly handle the situation when (e.g.,)} $\texttt{payment}$ is forbidden with a negative amount but must occur with a positive amount. This issue could prevent this approach from finding alignments that, instead, could be found when considering control flow and data in combination in the reference model.
In \cite{BurattinMS16}, the authors provide an algorithmic framework to efficiently check the conformance of Multi-Perspective Declare (MP-Declare) with respect to logs. \added{In this work, the authors reduce the data-aware conformance checking problem to a maximum constraint-satisfaction problem, where both data and model constraints are encoded. Alternatively, the alignment of data-aware declarative processes can be also reduced to a constraint satisfaction problem, where an optimization function is used to assess the alignment cost \cite{Borrego014}.  The two aforementioned works are, to the best of our knowledge, the only ones providing numerical characterization of the process conformance for declarative models.  Even though they provide a numerical approximation of the degree of conformance of a log trace against the model, no repair strategy is given. As we will observe in \S\ref{sec:dadtap}, the possibility of repairing traces is strictly related with the definition of strings and numerical data as elements of a partially ordered set, where missing values are the minimal elements of such a set. Furthermore, our work also extends the previous ones by also considering string data as well as numerical data.}