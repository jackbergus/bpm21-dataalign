\section{Related Work}
\label{sec:related}

Most of the conformance checking techniques reported in the scientific literature are based on procedural models. In \cite{DBLP:conf/edoc/AdriansyahDA11}, for the first time, the authors introduce conformance checking augmented with the notion of alignments. The work described in \cite{LeoniMA12} presents a (data agnostic) conformance checking approach based on the concept of alignment for declarative models.

Alignment-based approaches have also been used in \cite{MultiPerspective}, where the authors propose techniques for conformance checking with respect to data-aware procedural models. Since the latter approach first processes the control flow and then tests data conditions, it cannot be easily applied to the case of data-aware Declare. Indeed, in case the reference model contains constraints that are in conflict when considering the control flow in isolation (e.g., constraints requiring the existence and the absence of the same activity $\texttt{payment}$) this approach would stop after the first step because it cannot find any control flow-based alignment given the inconsistency of the model. However, the two constraints could be not in conflict when considering the data perspective, e.g., $\texttt{payment}$ is forbidden with a negative amount but must occur with a positive amount. This issue could prevent this approach from finding alignments that, instead, could be found when considering control flow and data in combination in the reference model.

More recently, in \cite{Borrego014}, the authors have presented an approach where the data perspective for conformance checking with Declare is expressed in terms of conditions on global process variables disconnected from the specific Declare constraints expressing the control flow. In \cite{BurattinMS16}, the authors provide an algorithmic framework to efficiently check the conformance of Multi-Perspective Declare (MP-Declare) with respect to event logs. These approaches do not provide repair strategies. 