\section{Approximate Constraint Matching}
The latter approach requires to enumerate all the possible model traces $\logtrace\in\TraceOf{\DeclModel}$: even if the constraints $P\in\Pi$ can be represented finitely, the generated traces can be both arbitrarily long and expensive to enumerate. Given that Business Process Management often handles finite process descriptions \cite{GiacomoV13} thus implying a finite interpretation of such formulae, we can represent each $P\in\Pi$ as an NFA \cite{GiacomoMM14}, which labels are logical propositions. As a next step, we can perform an approximate match of the resulting NFA with the trace $\nonlogtrace$ represented as a DAG \cite{Myers1989} by exploiting a variation of the well-known dynamic programming algorithm for string matching \cite{GALIL198833}. After extending the algorithm in \cite{Myers1989} by including the proposition evaluation as a part of the match \texttt{\color{red}[TODO]}, the outcome of such algorithm will be the $\logtrace\in\TraceOf{\DeclModel}$ providing the best match for $\nonlogtrace$ alongside its associated cost, $\mathcal{A}(\textup{NFA}_{P},\textup{DAG}_{\nonlogtrace})$. 
Please note that this approach can be exploited to rank the model constraints via $\mathcal{A}$:
\[\mathcal{R}_{\mathcal{A}}(P,\nonlogtrace)=\frac{1}{\mathcal{A}(\textup{NFA}_{P},\textup{DAG}_{\nonlogtrace})+1}\]