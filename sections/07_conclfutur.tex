\section{Conclusions}\label{sec:end}
In this paper, we presented an approach tackling conformance checking of log traces over data-aware Declare models. The proposed approach exploits Automated Planning for aligning the log traces and the reference model via a preliminary partitioning of the data space. The experiments show that the performance of the approach is acceptable even when the reference model contains a reasonably large number of data-aware constraints. In addition, since the implemented tool is independent by the planner used to solve the alignment problem, the improvements in the efficiency of the planners are automatically transferred to the tool.

Future works will investigate the relationship between planners and approximate path matching techniques \cite{Myers1989}, for the implementation of alignment approaches that return not only the optimal alignment but also suboptimal ones that might be of interest for the user. We will also investigate the possibility of performing alignments over data-aware knowledge bases \cite{10.1007/978-3-319-39696-5_18}, which potentially quicken the time required to test the satisfiability of the data conditions by conveniently indexing (i.e., pre-ordering) the payload space \cite{IdreosGNMMK12}. These approaches will still take advantage of the representation of the trace alignment problem as a planning problem. 