\documentclass{article}
	\def\papertitle{Aligning Data-Aware Declarative Process Models and Event Logs}
	\def\authors{G. Bergami, F. M. Maggi, A. Marrella, M. Montali}
	\def\journal{BPM'21}
% Define title defaults if not defined by user
\providecommand{\lettertitle}{Author Response to Reviews of}
\providecommand{\papertitle}{Title}
\providecommand{\authors}{Authors}
\providecommand{\journal}{Journal}
\providecommand{\doi}{--}

\usepackage[includeheadfoot,top=20mm, bottom=20mm, footskip=2.5cm]{geometry}

% Typography
\usepackage[T1]{fontenc}
\usepackage{times}
%\usepackage{mathptmx} % math also in times font
\usepackage{amssymb,amsmath}
\usepackage{microtype}
\usepackage[utf8]{inputenc}

% Misc
\usepackage{graphicx}
\usepackage[hidelinks]{hyperref} %textopdfstring from pandoc
\usepackage{soul} % Highlight using \hl{}

% Table

\usepackage{adjustbox} % center large tables across textwidth by surrounding tabular with \begin{adjustbox}{center}
\renewcommand{\arraystretch}{1.5} % enlarge spacing between rows
\usepackage{caption} 
\captionsetup[table]{skip=10pt} % enlarge spacing between caption and table

% Section styles

\usepackage{titlesec}
\titleformat{\section}{\normalfont\large}{\makebox[0pt][r]{\bf \thesection.\hspace{4mm}}}{0em}{\bfseries}
\titleformat{\subsection}{\normalfont}{\makebox[0pt][r]{\bf \thesubsection.\hspace{4mm}}}{0em}{\bfseries}
\titlespacing{\subsection}{0em}{1em}{-0.3em} % left before after

% Paragraph styles

\setlength{\parskip}{0.6\baselineskip}%
\setlength{\parindent}{0pt}%

% Quotation styles

\usepackage{framed}
\let\oldquote=\quote
\let\endoldquote=\endquote
\renewenvironment{quote}{\begin{fquote}\advance\leftmargini -2.4em\begin{oldquote}}{\end{oldquote}\end{fquote}}

\usepackage{xcolor}
\newenvironment{fquote}
  {\def\FrameCommand{
	\fboxsep=0.6em % box to text padding
	\fcolorbox{black}{white}}%
	% the "2" can be changed to make the box smaller
    \MakeFramed {\advance\hsize-2\width \FrameRestore}
    \begin{minipage}{\linewidth}
  }
  {\end{minipage}\endMakeFramed}

% Table styles

\let\oldtabular=\tabular
\let\endoldtabular=\endtabular
\renewenvironment{tabular}[1]{\begin{adjustbox}{center}\begin{oldtabular}{#1}}{\end{oldtabular}\end{adjustbox}}


% Shortcuts

%% Let textbf be both, bold and italic
%\DeclareTextFontCommand{\textbf}{\bfseries\em}

%% Add RC and AR to the left of a paragraph
%\def\RC{\makebox[0pt][r]{\bf RC:\hspace{4mm}}}
%\def\AR{\makebox[0pt][r]{AR:\hspace{4mm}}}

%% Define that \RC and \AR should start and format the whole paragraph 
\usepackage{suffix}
\long\def\RC#1\par{\makebox[0pt][r]{\bf RC:\hspace{4mm}}\textbf{\textit{#1}}\par} %\RC
\WithSuffix\long\def\RC*#1\par{\textbf{\textit{#1}}\par} %\RC*
\long\def\AR#1\par{\makebox[0pt][r]{AR:\hspace{10pt}}\textit{#1}\par} %\AR
\WithSuffix\long\def\AR*#1\par{\textit{#1}\par} %\AR*


%%%
%DIF PREAMBLE EXTENSION ADDED BY LATEXDIFF
%DIF UNDERLINE PREAMBLE %DIF PREAMBLE
\RequirePackage[normalem]{ulem} %DIF PREAMBLE
\RequirePackage{color}\definecolor{RED}{rgb}{1,0,0}\definecolor{BLUE}{rgb}{0,0,1} %DIF PREAMBLE
\providecommand{\DIFadd}[1]{{\protect\color{blue}\uwave{#1}}} %DIF PREAMBLE
\providecommand{\DIFdel}[1]{{\protect\color{red}\sout{#1}}}                      %DIF PREAMBLE
%DIF SAFE PREAMBLE %DIF PREAMBLE
\providecommand{\DIFaddbegin}{} %DIF PREAMBLE
\providecommand{\DIFaddend}{} %DIF PREAMBLE
\providecommand{\DIFdelbegin}{} %DIF PREAMBLE
\providecommand{\DIFdelend}{} %DIF PREAMBLE
%DIF FLOATSAFE PREAMBLE %DIF PREAMBLE
\providecommand{\DIFaddFL}[1]{\DIFadd{#1}} %DIF PREAMBLE
\providecommand{\DIFdelFL}[1]{\DIFdel{#1}} %DIF PREAMBLE
\providecommand{\DIFaddbeginFL}{} %DIF PREAMBLE
\providecommand{\DIFaddendFL}{} %DIF PREAMBLE
\providecommand{\DIFdelbeginFL}{} %DIF PREAMBLE
\providecommand{\DIFdelendFL}{} %DIF PREAMBLE
%DIF END PREAMBLE EXTENSION ADDED BY LATEXDIFF


\begin{document}

% Make title
{\Large\bf \lettertitle}\\[1em]
{\huge \papertitle}\\[1em]
{\authors}\\
{\it \journal, }\texttt{doi:\doi}\\
\hrule

% Legend
\hfill {\bfseries RC:} \textbf{\textit{Reviewer Comment}},\(\quad\) AR: \emph{Author Response}, \(\quad\square\) Manuscript text


\section{Reviewer \#1}

\subsection{On the usage of Automated Planning}\label{sec:uap}

\RC What I do not fully understand is the need to use Automated Planning techniques here. It seems that the alignment problem on the automatons is not really different from the well-known alignment problem, using synchronous moves, log moves (inserts in this paper), and model moves (deletes). As such, it seems that the paper introduces two ideas that seem orthogonal: How can we align data-aware declarative models to event logs, and how can Automated Planning techniques be used to find alignments. As such, it feels that this paper lacks a single focus. As a result, if the paper would be accepted, I believe that the authors should make clear why the known alignment techniques do not work here. If these do work, the section on the Automated Planning is very odd. To me, it seems like the paper should be split into two papers.

\AR [Comment by Marco, explanation by Andrea in the Paper]

\subsection{Minor Comments}

\RC p1. "in the context data-aware" => "in the context of data-aware".

\AR Done

%%%
\RC p.2: LTL\_f appears 'out of the blue'. Please introduce this before.

\RC p.6: Here, LTL\_f is introduced, which is too late.

\AR As suggested, we gave a brief description of LTL\_f in the introduction.

%%%
\RC p.5: "on the footsteps" => either "in the footsteps" or "on the footpath".

\AR Done

%%%
\RC p.6: "such that $\sigma_i \vDash t_i$": Unclear, what does this mean?

\AR It means that any trace composed of events $\sigma_i$ can be expressed as a finite sequence of predicates $t_i$ such that $t_i$ is satisfied by $\sigma_i$. The notation is coming from [7] and an explanation of how to do that is given at \S 5.1. 

%%%
\RC Ex.1 (continued), last line: This seems incorrect. The F clause now also accepts if, say, B.x = 1, whereas the original F clause did not. If guess the or-clause should be in the G clause (and not in this F-clause): G(not(C) $\vee$ F(B $\wedge$ (0 < B.x <= 3) $\vee$ B.x > 3))). This matches the decomposition into the three intervals much better.

\AR The reviewer addresses here the outcome of the decomposition which, still, is achieved in multiple subsequent steps. Still, the proposed decomposition is only partial, as it not considers the $y$ variable. Please refer to the subsequent parts of the Example 1 for the final decomposition and atomization producing the $\Sigma$ required by the alignments. 

%%%
\RC p.10: A delete operation corresponds to a move on model, whereas an insert operation corresponds to a move on log. This suggests that we now have the standard problem of aligning a trace on a model, and could use the standard alignment techniques. Is this correct? If so, using the planner as suggested is an option, but standard alignment techniques could also be used, right?

\AR On the usage of standard alignment techniques vs. planners, please see Section \ref{sec:uap}. This also relates to the previous observation from the reviewer, i.e.:


%%%
\RC p.9: [i]t seems replacements are not used to augment the trace (see next page). Why introduce them here?

In this section, we introduce the notation we exploit for the repair sequences generated by the PDDL. Despite replacements might be seen as a syntactic sugar for deletions immediately followed by insertions, the definition of such a constraints has direct implications on the final outcome of the alignment: as we observed in the paper, replacement operations might be favored instead of insertions or deletions by assigning to the latter an inferior cost. In fact, the usage of only insertions or deletions will only favor single insertions or deletions operations, while rearing one single event trace might be perceived as less costly than inserting or removing events from a trace. Last, this would require to restrict our analysis to alignment techniques supporting replacement costs as in Levensthein measures. 


%%%
\RC Def.1: Unclear what is being defined here. Reads more like a theorem. Given a trace t and and a model M, a repair sequence r is a sequence of operations (deletion, insertion, replacement) such that r(t) conforms to M. The repair sequence r is called optimal if there exists no repair sequence r' such that the costs of r' are lower than the costs of r. The conformance checking problem is to find an optimal repair sequence for trace t and model M. (?)

\AR As stated in the previous line, we provided a formal definition of the conformance checking problem. Thus, we motivate the required formalisms that will follow. 

%%%
\RC Def.1: "it exists" => "there exists".

\AR Done

%%%
\RC Fig.4: The arc from s\_5 to s\_6 is labeled p\_7. I think this should be p\_8.

\AR Done

%\begin{quote}
%The cat in the box is \DIFdelbegin \DIFdel{dead}\DIFdelend \DIFaddbegin \DIFadd{alive}\DIFaddend .
%\begin{align}
%E &= mc^2 \\
%m\cdot \DIFdelbegin \DIFdel{a=F}\DIFdelend \DIFaddbegin \DIFadd{v=p}\DIFaddend .
%\end{align}
%\end{quote}
%
%\AR*But I actually have no idea what you were talking about.

\end{document}