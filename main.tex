\RequirePackage{fix-cm}
\documentclass[runningheads]{llncs}

\let\proof\relax
\let\endproof\relax
\usepackage{amsthm}
\newtheorem{ex}[theorem]{Example}
\newtheorem*{excont}{Example \continuation}
\newcommand{\continuation}{??}
\newenvironment{continueexample}[1]
{\renewcommand{\continuation}{\ref{#1}}\excont[continued]}
{\endexcont}

\usepackage{graphicx}
\usepackage{xcolor}
\usepackage{hyperref}
\usepackage[inline]{enumitem}
\usepackage{braket}
\usepackage{caption}
\usepackage{subcaption}
\usepackage{booktabs}
\usepackage{xspace}

\usepackage{comment}
\usepackage{tikz}
\usepackage{pgfplots}
\usepackage{pgfplotstable}

% Used for displaying a sample figure. If possible, figure files should
% be included in EPS format.
%
% If you use the hyperref package, please uncomment the following line
% to display URLs in blue roman font according to Springer's eBook style:
\renewcommand\UrlFont{\color{blue}\rmfamily}

\newcommand{\Sdeclare}[3]{\textsf{#1(}\texttt{#2},\;\texttt{#3}\textsf{)}}
\newcommand{\Pfdeclare}[4]{\ensuremath{\textsf{#1(}\texttt{#2},\;\texttt{#3},\;#4\textsf{)}}}
\usepackage{amsmath,scalerel,amssymb}
\newcommand{\Globally}{\ensuremath{\square}}
\newcommand{\Finally}{\ensuremath{\lozenge}}
\newcommand{\Next}{{\ensuremath\raisebox{0.25ex}{\text{\scriptsize$\bigcirc$}}}}
\newcommand{\Until}{\ensuremath{\mathbin{\mathcal{U}}}}
\newcommand{\Release}{\ensuremath{\mathbin{\mathcal{R}}}}
\newcommand{\Wntil}{\ensuremath{\mathbin{\mathcal{W}}}}

\usepackage{xparse}
\NewDocumentCommand{\INTERVALINNARDS}{ m m }{
	#1 {,} #2
}
\NewDocumentCommand{\interval}{ s m >{\SplitArgument{1}{,}}m m o }{
	\IfBooleanTF{#1}{
		\left#2 \INTERVALINNARDS #3 \right#4
	}{
		\IfValueTF{#5}{
			#5{#2} \INTERVALINNARDS #3 #5{#4}
		}{
			#2 \INTERVALINNARDS #3 #4
		}
	}
}
\makeatletter
\newcommand\xleftrightarrow[2][]{%
	\ext@arrow 9999{\longleftrightarrowfill@}{#1}{#2}}
\newcommand\longleftrightarrowfill@{%
	\arrowfill@\leftarrow\relbar\rightarrow}
\makeatother
\DeclareMathSizes{10}{10}{7}{5}


%%%%%%%%%%%%%%%%%%%%%%%%%%%%%%%%%%%%%%%%%%%%%%%%%%%%%%%%%%%%
%% TIKZ CONFIGURATION
%%
%% Sebastian Sardina - ssardina@gmail.com
%%%%%%%%%%%%%%%%%%%%%%%%%%%%%%%%%%%%%%%%%%%%%%%%%%%%%%%%%%%%%%


\usetikzlibrary{petri,arrows,snakes,backgrounds,matrix,automata,shapes,shadows,patterns,fit,calc}

%%%% PICTURES AND PETRI NETS CONFIGURATIONS
\tikzstyle{every picture}=[->,>=stealth',shorten >=1pt,auto,node distance=1.3cm,semithick]

\tikzstyle{place}=[circle,thick,draw=black,minimum size=4mm]
\tikzstyle{invisible place}=[place,draw=none,fill=none]
\tikzstyle{transition}=[rectangle,inner ysep=1,thick,draw=black!75,fill=black!10,minimum size=2mm,,minimum width=4mm]
\tikzstyle{itransition}=[transition,draw=none,fill=none]
  \tikzstyle{every label}=[black]


%%%%%%%%%%
%% EXTERNALIZATION
%%%%%%%%%%
\usetikzlibrary{external}
\tikzsetexternalprefix{externalfigs/}
% \tikzexternalize
% \tikzexternalize[mode=list and make]

  % \tikzexternalize % activate!
\newcommand{\tikzsetforce}{\tikzset{external/force remake}}
\newcommand{\tikzsetforcenext}{\tikzset{external/remake next}}


% Configuration to externlize overlays (a must when using beamer)
\makeatletter
\AtBeginDocument{\@ifpackageloaded{beamer}{%
	\newcommand*{\overlaynumber}{\number\beamer@slideinframe}
	\tikzset{
	  beamer externalizing/.style={%
	    execute at end picture={%
	      \tikzifexternalizing{%
	        \ifbeamer@anotherslide
	        \pgfexternalstorecommand{\string\global\string\beamer@anotherslidetrue}%
	        \fi
	      }{}%
	    }%
	  },
	  external/optimize=false
	}
	\let\orig@tikzsetnextfilename=\tikzsetnextfilename
	\renewcommand\tikzsetnextfilename[1]{\orig@tikzsetnextfilename{#1-\overlaynumber}}
	\tikzset{every picture/.style={beamer externalizing}}
}
{\relax}
}
\makeatother
%
%


\begin{document}
%
\title{Aligning Data-Aware Declarative Process Models and Event Logs}
%
%\titlerunning{Abbreviated paper title}
% If the paper title is too long for the running head, you can set
% an abbreviated paper title here
%
\author{Giacomo Bergami\inst{1} \and
Fabrizio Maria Maggi\inst{1} \and
\\Andrea Marrella\inst{2} \and Marco Montali\inst{1}}

%
\authorrunning{G. Bergami et al.}
% First names are abbreviated in the running head.
% If there are more than two authors, 'et al.' is used.
%
\institute{Free University of Bozen-Bolzano, Italy \\
\email{gibergami@unibz.it, \{maggi,montali\}@inf.unibz.it}\\
 \and
Sapienza - University of Rome, Italy\\
\email{marrella@diag.uniroma1.it}}
%
\maketitle              % typeset the header of the contribution
\linespread{0.98}
%
\begin{abstract}
Alignments are a conformance checking strategy quantifying the amount of deviations of a trace with respect to a process model, as well as providing optimal repairs for making the trace compliant with the process model. Data-aware alignment strategies are also gaining momentum, as they provide richer descriptions for deviance detection. Nonetheless, no technique is currently able to provide trace repair solutions in the context data-aware declarative process models: current approaches either focus on procedural models, or numerically quantify the deviance with no proposed repair strategy. After discussing our working hypotheses, we show how such a problem can be reduced to a data agnostic trace alignment problem, while ensuring the correctness of its solution. Finally, we show how to find such a solution, i.e., how to align traces with data-aware declarative models by adding/deleting events in the trace or by changing the attribute values attached to them.

\keywords{Conformance Checking  \and Alignments \and  Data-Aware Declarative Models.}
\end{abstract}

\section{Introduction}
\label{sec:introduction}

\textit{Conformance checking} is a branch of process mining assessing whether a sequence of distinguishable events (i.e., a \textit{trace}) conforms to the expected process behavior represented as a \textit{process model} \cite{RozinatA08}. When a trace does not conform to the model, we say that the trace is \textit{deviant}. In this case, techniques based on cost-driven alignments additionally provide minimal repair strategies to make the trace conformant to the model \cite{DBLP:conf/edoc/AdriansyahDA11}. Alignments represent a valuable instrument for business analysts, as the combined provision of alternative repair strategies, ranked by alignment cost, supports the business analyst in choosing among different process improvement strategies. In conformance checking, models can be described by either procedural or declarative languages;  while the former fully enumerate the set of all the possible allowed traces, the latter %\deleted{provide a compact process representation by}
list the constraints delimiting the expected behavior. %Nevertheless, %\deleted{LTL$_f$-based}
\added{Declarative process models like Declare models \cite{DBLP:conf/edoc/PesicSA07}, whose semantics can be expressed in \added{Linear Time Logic on finite traces (LTL$_f$)} \cite{GiacomoV13} can always be transformed into constraint automata.}
%\added{As LTL$_f$ is an extension of modal logic in which worlds are organized in an finite linear structure, this logic is well suited to describe business processes logs having traces of finite length \cite{GiacomoV13}.}
%
%In fact, such semantics describes the actions that will follow when some pre-conditions are met \cite{LiPZVR20}.
The representation of Declare models as automata can be adopted for aligning traces with this type of models \cite{LeoniMA12,XuLZ17a}.
\\
\indent
Multi-perspective checking for process conformance is gaining momentum, as conformance checking techniques considering both trace types and data annotations as ``first-class citizens'' enable to discover more deviations \cite{MultiPerspective}. This reflects the essence of real-world business processes, which are inherently described by both processes and their different domain objects \cite{PetermannJMR14} (e.g., employees, products, etc.), which can be encoded as traces and event data. While alignment-based  data-aware conformance has been already investigated in the context of procedural models, most of the conformance checking approaches for data-aware declarative models \cite{BurattinMS16,Borrego014} focus on a numerical approximation of the degree of conformance of a trace against the model and do not provide repair strategies.
\\
\indent
To tackle this research gap, we propose a novel approach for aligning event logs to data-aware declarative models by reducing it to a data-agnostic alignment problem over LTL$_f$-based models. This solution exploits the following considerations: \begin{enumerate*}[label=\emph{\alph*})]
	\item \label{it1} to represent the process model, we use a sub-set of the data-aware extension of Declare presented in \cite{BurattinMS16}. After representing the data-aware Declare model using a data\added{-}agnostic LTL$_f$ semantics,
	\item \label{it2} we exploit the data predicates in the data-aware Declare clauses to partition the data space. This provides propositions representing data in addition to event labels. Then,
	\item we combine each event label with the propositions generated in \ref{it2} and transform the model in \ref{it1} into its data-aware counterpart. The automata-based representation of such a model is used to align traces (seen as sequences of events with a payload of data attribute-value pairs) with the model.
In particular, we show that the alignment problem can be expressed as a planning problem in Artificial Intelligence, which can be efficiently solved by selected state-of-the-art planners \cite{XuLZ17a,Marrella17}.
%
\\
\indent
\added{Despite the resulting data-agnostic alignment via planning is semantically equivalent to customary cost-based aligners \cite{DBLP:conf/edoc/AdriansyahDA11}, our previous work \cite{XuLZ17a} showed that the former outperforms the latter in terms of computational performance and scalability in the presence of models of considerable size, which is the case of this paper. In fact, as a consequence of the reduction of the data-aware alignment problem into a data-agnostic one, the automata-based process models used as input for our approach have several more transitions and states than in traditional alignment problems. Therefore, as we needed to show the feasibility of our approach, we decided to resort to planning-based alignments for both presenting our framework outline and performing the experiments.}
%
%\added{According to our previous works, \cite{XuLZ17a,LeoniM17} already remarked that such algorithms outperform state-of-the-art trace alignment algorithms where data payload is not considered} .
%
\added{Planners generate} repair strategies able to align traces with a data-aware declarative model based on changes at the level of control flow (such as adding/deleting events) or at the level of the data flow (such as changing the attribute values attached to them).
\end{enumerate*}

%The rest of the paper is structured as follows: after providing relevant related works (\S\ref{sec:related}), we introduce the notion of event log (\S\ref{ssec:elog}) and the data-aware declarative language used to represent the model (\S\ref{ssec:dad}); we also provide hints on Automated Planning, as we will later exploit the SymBA*-2 optimal planner \cite{torralba2014symba} to compute the alignments (\S\ref{ssec:ap}). These preliminary notions guide us into the definition of our  working assumptions adhering to the literature of reference (\S\ref{sec:wa}). After deep-diving into the technical details providing the solution to the data-aware declarative alignment problem (\S\ref{sec:dccap}), we benchmark SymBA*-2 over a synthetic dataset and discuss its performance in this context (\S\ref{sec:experiments}). Last, we draw our final conclusions and propose some future work (\S\ref{sec:end}).


%we draw the working assumptions jointly with the assumptions from the current literature on declarative conformance checking (\S\ref{sec:wa}). After outlining the declarative model alignment over which we want to reduce the problem (\S\ref{sec:dccap}), we deep-dive into the data-aware Declare Trace Alignment Pipeline (\S\ref{sec:dadtap}) \texttt{\color{red} [TODO: experiments? ]} Last, we draw our final conclusions and propose some future works (\S\texttt{\color{red}[TODO]}).

\section{Related Work}
\label{sec:related}

Most of the conformance checking techniques reported in the scientific literature are based on procedural models. In \cite{DBLP:conf/edoc/AdriansyahDA11}, for the first time, the authors introduce conformance checking augmented with the notion of alignments. The work described in \cite{LeoniMA12} presents a (data agnostic) conformance checking approach based on the concept of alignment for declarative models.

Alignment-based approaches have also been used in \cite{MultiPerspective}, where the authors propose techniques for conformance checking with respect to data-aware procedural models. Since the latter approach first processes the control flow and then tests data conditions, it cannot be easily applied to the case of data-aware Declare. Indeed, in case the reference model contains constraints that are in conflict when considering the control flow in isolation (e.g., constraints requiring the existence and the absence of the same activity $\texttt{payment}$) this approach would stop after the first step because it cannot find any control flow-based alignment given the inconsistency of the model. However, the two constraints could be not in conflict when considering the data perspective, e.g., $\texttt{payment}$ is forbidden with a negative amount but must occur with a positive amount. This issue could prevent this approach from finding alignments that, instead, could be found when considering control flow and data in combination in the reference model.

More recently, in \cite{Borrego014}, the authors have presented an approach where the data perspective for conformance checking with Declare is expressed in terms of conditions on global process variables disconnected from the specific Declare constraints expressing the control flow. In \cite{BurattinMS16}, the authors provide an algorithmic framework to efficiently check the conformance of Multi-Perspective Declare (MP-Declare) with respect to event logs. These approaches do not provide repair strategies. 
\newcommand{\lnext}{\ensuremath{\mathbf{X}}}
\newcommand{\lwnext}{\ensuremath{\mathbf{\bar{X}}}}
\newcommand{\luntil}{\ensuremath{\mathbf{U}}}
\newcommand{\lsince}{\ensuremath{\mathbf{S}}}
\newcommand{\lrelease}{\ensuremath{\mathbf{R}}}
\newcommand{\lwuntil}{\ensuremath{\mathbf{W}}}
\newcommand{\lglobally}{\ensuremath{\mathbf{G}}}
\newcommand{\lfuture}{\ensuremath{\mathbf{F}}}
\newcommand{\tnext}{\ensuremath{\mathbf{X}_{I}}}
\newcommand{\twnext}{\ensuremath{\mathbf{\bar{X}_I}}}
\newcommand{\tuntil}{\ensuremath{\mathbf{U}_{I}}}
\newcommand{\tsince}{\ensuremath{\mathbf{S}_{I}}}
\newcommand{\trelease}{\ensuremath{\mathbf{R}_{I}}}
\newcommand{\tglobally}{\ensuremath{\mathbf{G}_{I}}}
\newcommand{\lonce}{\ensuremath{\mathbf{O}}}
\newcommand{\tonce}{\ensuremath{\mathbf{O}_{I}}}

\newcommand{\lyesterday}{\ensuremath{\mathbf{Y}}}
\newcommand{\tyesterday}{\ensuremath{\mathbf{Y}_{I}}}
\newcommand{\lhistorically}{\ensuremath{\mathbf{H}}}
\newcommand{\thistorically}{\ensuremath{\mathbf{H}_{I}}}
\newcommand{\tfuture}{\ensuremath{\mathbf{F}_{I}}}

%\newcommand{\n}{\ensuremath{\figitem{N}}}}
%\newcommand{\x}{\ensuremath{\figitem{X}}}
%\newcommand{\y}{\ensuremath{\bar{\textsf{X}}}}
%\newcommand{\R}{\ensuremath{\mathbf{R}_+}}


\section{Background}
%\section{Modeling Data-Aware Declare Alignment}
%In this section, we introduce the requirements enabling the computation of data-aware declare alignments.

\subsection{Event Logs}
(Data) \textit{payloads} are finite functions $p\in V^K$, where $K$ is a finite set of keys and $V$ is a (finite) set of data values. We consider also the case in which the value of a certain key $k$ is missing in a payload. In particular, we denote as $\varepsilon$ an element $\varepsilon\notin V$, such that $p(k)=\varepsilon$ for $k\notin\textup{dom}(p)$. Given a finite set of activity labels $\textsf{Act}$, an event $\sigma_j$ is a pair $\Braket{\texttt{A},p}$, where $\texttt{A}\in\textsf{Act}$ is an activity label, and $p$ is a payload; we denote with $\lambda$ (and $\varsigma$) the first (and second) projection of such pair, i.e., $\lambda(\sigma_j)=\texttt{A}$ (and $\varsigma(\sigma_j)=p$). A \textit{trace} $\sigma$ is a temporally-ordered and finite sequence of distinct events $\sigma_1\cdots\sigma_n$, modeling a process run. We distinguish the trace keys ($K_t$) from the event keys ($K_e$), such that $K=K_t\cup K_e$ with $K_t\cap K_e=\emptyset$: all events within the same trace associate the same values to the same trace keys, i.e., $\forall \Braket{\texttt{A}_i,p_i},\Braket{\texttt{A}_j,p_j}\in\sigma.\;\forall k\in K_t.\; p_i(k)=p_j(k)$. A log $\mathcal{L}$ is a finite set of traces. This  characterization is compliant with the \textsc{eXtensible Event Stream} format, which is the \textit{de facto} standard for representing event logs within the Business Process Management community \cite{XES}.


\subsection{Data-Aware Declare}\label{ssec:dad}
Declare is a declarative process modeling language. A Declare model $\mathcal{M}$ is described as a set of constraints $\Set{c_1,\dots,c_m}$ that must be simultaneously satisfied throughout a process execution.
%Due to space limitations, we avoid providing a detailed description of Data-Aware Declare \cite{}, henceforth simply referred as Declare, and we will only describe its general features.
Such constraints express either positive (or negative) dependencies between two events having labels in $\textsf{Act}$, or quantify the occurrence of events having a specific label in $\textsf{Act}$. In the first case, one of the two clause labels is called \textit{activation}, and the other \textit{target}; while testing a trace $\sigma$ for conformance over this clause, the presence of the activation label in $\sigma$ triggers the clause verification, requiring the (non-)execution of an event containing the target label in the same trace.

Declare has been extended to include conditions over data in the Declare constraints \cite{BurattinMS16}. In the context of this paper, we will consider two types of data predicates $\phi^d$ (\textit{conditions}) decorating activations (a.k.a.\ activation conditions) and targets (a.k.a.\ target conditions), respectively.
%The conditions over the activation (or target) labels \texttt{A} can be expressed via predicates $\texttt{A}$ \cite{SchonigCMM16,LenoDM18}.
While activation conditions must be valid when an event exhibiting the activation label occurs, target conditions impose value limitations on the payload of events containing the target label.


We use atom $\texttt{A}$ as a shorthand for $\lambda(\sigma_i)=\texttt{A}$ for each $\texttt{A}\in\textsf{Act}$ given an event $\sigma_i$ to be assessed, while $\phi^d$ is a propositional formula containing as atoms either the universal truth ($\top$), or the falsehood ($\bot$), or a binary relation ``$\texttt{A}.k\;\Re\;c$'', where $c$ is a constant value representing either a number or a string, $\Re$ is either an equality or a precedence/subsequent relation over values in $V$ or their negation, and $k\in K$ acts as a placeholder for $\varsigma(\sigma_i)(k)$, where $\varsigma(\sigma_i)$ is the payload associated to the event $\sigma_i$ and $k$ is associated to a value $\sigma(\sigma_i)(k)$. E.g., ``$\texttt{RP}.\textit{quality}\leq 3$'' is formally represented as $\varsigma(\sigma_i)(k)\leq 3$ for any event $\sigma_i$ having $\lambda(\sigma_i)=\texttt{RP}$.  This is a widely adopted assumption, that spans from data-aware procedural models \cite{MultiPerspective} to data-aware declarative models \cite{10.1007/978-3-642-40176-3_8}. Furthermore, this assumption can be also adapted to categorical data, as strings are ordered via lexicographical orderings over the single characters \cite{MultiPerspective}. We denote the \textit{compound conditions}, namely the conjunction of label requirements and data conditions, as $\psi=\texttt{A}\wedge \phi^d$.



The semantics of the Declare constraints we consider here is represented in \tablename~\ref{tbl:timed-mfotl}.
Here, the $\lfuture$, $\lnext$, $\lglobally$, and $\luntil$ LTL$_f$ future operators have the following meanings: formula $\lfuture \psi_1$ means that $\psi_1$ holds sometime in the future, $\lnext \psi_1$ means that $\psi_1$
holds in the next position, $\lglobally \psi_1$ says that $\psi_1$ holds forever in the future, and, lastly, $\psi_1 \luntil \psi_2$ means that sometime in the future $\psi_2$ will hold and
until that moment $\psi_1$ holds (with $\psi_1$ and $\psi_2$ LTL$_f$ formulas).
The $\lonce$, $\lyesterday$ and $\lsince$ LTL$_f$ past operators have the following meaning:
$\lonce \psi_1$ means that $\psi_1$ holds sometime in the past,
$\lyesterday \psi_1$ means that $\psi_1$ holds in the previous position,
and $\psi_1 \lsince \psi_2$ means that $\psi_1$ has held sometime in the past and since that moment $\psi_2$ holds.

\begin{table*}[t!]
\caption{Semantics for MP-Declare\ constraints in LTL$_f$. \label{tbl:timed-mfotl}}
\centering
\scriptsize{
\begin{tabular}{ll}
\toprule
\textbf{Template} & \textbf{LTL$_f$ Semantics} \\
\midrule
existence & $\top \rightarrow \lfuture (\texttt{A} \textcolor{gray}{\wedge \phi^d}) \vee \lonce (\texttt{A} \textcolor{gray}{\wedge \phi^d}))$ \\
\midrule
responded existence  & $\lglobally( (\texttt{A} \textcolor{gray}{\wedge \phi^d}) \rightarrow (\lonce (\texttt{B} \textcolor{gray}{\wedge \phi^d}) \vee \lfuture (\texttt{B} \textcolor{gray}{\wedge \phi^d})))$ \\
\midrule
response &  $\lglobally(  (\texttt{A} \textcolor{gray}{\wedge \phi^d}) \rightarrow \lfuture (\texttt{B} \textcolor{gray}{\wedge \phi^d}))$ \\
alternate response  & $ \lglobally( (\texttt{A} \textcolor{gray}{\wedge \phi^d}) \rightarrow \lnext(\neg (\texttt{A} \textcolor{gray}{\wedge \phi^d}) \luntil (\texttt{B} \textcolor{gray}{\wedge \phi^d}))$ \\
chain response &  $\lglobally( (\texttt{A} \textcolor{gray}{\wedge \phi^d}) \rightarrow \lnext (\texttt{B} \textcolor{gray}{\wedge \phi^d}))$ \\
\midrule
precedence &  $\lglobally( (\texttt{B} \textcolor{gray}{\wedge \phi^d}) \rightarrow \lonce (\texttt{A} \textcolor{gray}{\wedge \phi^d}))$ \\
alternate precedence & $ \lglobally( (\texttt{B} \textcolor{gray}{\wedge \phi^d}) \rightarrow \lyesterday(\neg (\texttt{B} \textcolor{gray}{\wedge \phi^d}) \lsince (\texttt{A} \textcolor{gray}{\wedge \phi^d}))$ \\
chain precedence & $\lglobally( (\texttt{B} \textcolor{gray}{\wedge \phi^d}) \rightarrow \lyesterday (\texttt{A} \textcolor{gray}{\wedge \phi^d}))$ \\
\midrule
not responded existence  &
$\lglobally( (\texttt{A} \textcolor{gray}{\wedge \phi^d}) \rightarrow \neg (\lonce (\texttt{B}   \textcolor{gray}{\wedge \phi^d}) \vee \lfuture (\texttt{B} \textcolor{gray}{\wedge \phi^d})))$ \\
not response  & $\lglobally(  (\texttt{A} \textcolor{gray}{\wedge \phi^d}) \rightarrow \neg \lfuture (\texttt{B} \textcolor{gray}{\wedge \phi^d}))$ \\
not precedence & $\lglobally( (\texttt{B} \textcolor{gray}{\wedge \phi^d}) \rightarrow \neg \lonce (\texttt{A} \textcolor{gray}{\wedge \phi^d}))$ \\
not chain response  & $\lglobally( (\texttt{A} \textcolor{gray}{\wedge \phi^d}) \rightarrow \neg \lnext (\texttt{B} \textcolor{gray}{\wedge \phi^d}))$ \\
not chain precedence  & $\lglobally( (\texttt{B} \textcolor{gray}{\wedge \phi^d}) \rightarrow \neg \lyesterday (\texttt{A} \textcolor{gray}{\wedge \phi^d}))$ \\
\bottomrule
\end{tabular}}
\end{table*}





%Even if current literature  considers \textit{correlation} conditions between activations and targets \cite{SchonigCMM16}, we will not model such
%constraints as previously discussed in \S\ref{sec:mot}. Such constraints can be represented by either an intuitive graphical representation, which makes them easy to use and interpret for process analysts, or with a formal semantics \cite{LeoniMA12}.

\subsection{Automated Planning}\label{ssec:ap}
Planning systems are problem-solving algorithms, modeling a problem as a set of possible configurations that can be reached through a sequence of actions \cite{APlan}. Using planning systems such as PDDL, it is possible to formulate such problems as $\mathcal{P}=(I,G,\mathcal{P}_\mathcal
{D})$, where $I$ is the description of the initial world configuration, $G$ is the goal configuration, and $\mathcal{P}_\mathcal{D}$ is the planning domain. The domain is built upon a set of propositions describing the state of the world (i.e., the set of valid propositions) and a set of actions $\Omega$ that can be performed. An action schema $a\in \Omega$ is in the form $a=\Braket{\textit{Par}_a,\textit{Pre}_a,\textit{Eff}_a}$, where $\textit{Par}_a$ is the list of the input parameters for $a$, $\textit{Pre}_a$ defines the preconditions under which $a$ can be performed, and $\textit{Eff}_a$ specifies the effects of the action on the current world configuration. Both $\textit{Pre}_a$ and $\textit{Eff}_a$ are represented as propositions in $\mathcal{P}_\mathcal{D}$ via boolean predicates and numeric fluents.

Recently, the planning community has developed several planners implementing scalable search heuristics, which enable the solution of challenging problems in several Computer Science domains \cite{Marrella17}. Walking on the footsteps of \cite{MaggiMCA18}, we focus on planning techniques characterized by fully observable and static domains providing a perfect world description. In these scenarios, a sequence of actions whose execution transforms the initial state into a state satisfying the goal is the desired solution. In order to represent numeric alignment costs, we exploit the STRIPS fragment of PDDL, thus keeping track of the costs of planning actions and synthesizing plans satisfying pre-specified metrics.

\begin{figure}[!t]
	\centering
%	\begin{subfigure}[b]{0.45\textwidth}
%		\centering
%		\includegraphics[width=\textwidth]{images/example_1_graph}
%		\caption{$\bot\Release(\neg p_3\vee p_1\vee p_2) \wedge \top\mathcal{U}p_1$}
%		\label{fig:g1}
%	\end{subfigure}
%	\hfill
%	\begin{subfigure}[b]{0.45\textwidth}
%
%	\end{subfigure}
\includegraphics[scale=0.9]{images/example_3_graph}
%\caption{$\bot\Release(\neg\texttt{A}\vee(\top\Until(p_2\vee p_3)))\wedge\top\Until p_3$}
%\label{fig:g2}
	\caption{Representation of the LTL$_f$ formula $\lglobally(\neg\texttt{A}\vee(\lfuture(p_4\vee p_5\vee p_6\vee p_7\vee p_8\vee p_9)))\wedge \lfuture p_8$  as a constraint automaton \cite{Westergaard11}, where $\Sigma$ contains all the non-$\bot$ and non-$\top$ atoms.}
	\label{fig:g1g2}
\end{figure}

\section{Working Assumptions}\label{sec:wa}

In this section, we outline some working assumptions that can be inferred from the literature of reference. First, we assume that \begin{enumerate*}[label=\emph{\alph*})]
\item compliance requirements of Declare models can be expressed in a formal language such as Linear Time Logic on Finite Traces (LTL$_f$) \cite{10.1007/978-3-642-40176-3_8}, as real-world logs contain only traces of finite length \cite{GiacomoV13};
\item we restrict the space of the possible alignments of the log trace repairs to the traces generated by the automaton representation of the Declare model;
%\item as in \cite{XuLZ17a}, we want to align a single log trace against a declare model, thus restricting the space of the possible alignments;
\item differently from \cite{MultiPerspective}, we can avoid to model reading and writing operation, as the entirety of our analysis will be conducted \textit{post-mortem}; \item last, each event trace must be represented by one single proposition: similarly to the non-data aware scenario \cite{XuLZ17a}, each event trace should be associated to just one activity label. \end{enumerate*} As we will see in the incoming section, the latter consideration will require us to partition the possible data space into distinct propositions.


%Still, we can freely assume that the constraint automatons generated from the LTL$_f$ interpretation of such Declare constraints allow to represent any possible event label that is not represented within the Declare constraints, by either representing it as a transition $\Sigma\backslash S$, where $\Sigma$ is the set of all the possible strings and $S$ is a (possibly empty) finite set of traces that we want to ignore \cite{LeoniMA12,Westergaard11}, or by representing it as finite conjunction of negated predicates \cite{Lydia}, where each predicate is a proposition that can be deduced from a Declare Model represented in LTL$_f$. Figure~\ref{fig:g1g2} provides a intuitive representation of some LTL$_f$ formulae in the former representation.


Given an appropriately chosen set $\Sigma$ of atoms, it is always possible to represent a trace $\sigma=\sigma_1\cdots \sigma_n$ as a finite sequence $t_\sigma=t_1\cdots t_n$, where, for $1\leq i\leq n$, $t_i$ is a unique atom $t_i\in\Sigma$ such that $\sigma_i\vDash t_i$ \cite{XuLZ17a,MaggiMCA18}.
Contextually,  any LTL$_f$ formula $\varphi_{\mathcal{M}}$ representing a Declare model $\mathcal{M}$ can be represented as a deterministic finite-state automaton (DFA) $\mathcal{A}_{\varphi_{\tiny\mathcal{M}}}$ \cite{Westergaard11,Lydia} accepting all the sequences $t_\sigma$ from traces $\sigma$ satisfying $\varphi_{\mathcal{M}}$ (see Figure~\ref{fig:g1g2}). A DFA  $(\Sigma,Q,q_0,\rho,F)$ is defined \cite{0016921} over a finite set of states $Q$ reading as an input symbols from a finite alphabet $\Sigma$ that are consumed by traversing the automaton from a starting state $q_0\in Q$ via a transition function $\rho\colon Q\times \Sigma\to Q$; the input sequence is accepted once the input sequence is completely digested and an accepting state in $F\subseteq Q$ is reached through navigation. Given that in the non data-aware Declare scenario the atoms within LTL$_f$ could be either $\top$, or $\bot$, or $\psi=\texttt{A}$, $\Sigma$ corresponds to the activity set  $\textsf{Act}$, as each event is associated to one single label. For data-aware Declare we will extend $\Sigma$ o take into consideration propositional formulas representing data conditions.

Last, we freely assume that all the events having the same label will always contain the same set of keys, with possibly differently associated values. This is a common assumption in the relational database field, where all the rows belonging to the same table contain the same number of values. We also freely assume that missing values are represented with specific values, such as an empty string, $-1$, $0$, $-\infty$, or $+\infty$, depending on the context. 
\section{Data-Aware Declarative Conformance Checking as Planning}\label{sec:dccap}
In this section, we study the problem of aligning log traces $\sigma\in\mathcal{L}$ and a (data-aware) Declare model $\mathcal{M}$ for data-aware declarative conformance checking: to do so, we firstly reduce such problem to a mere automaton sequence acceptation task via a specific set of atoms $\Sigma$ (\textit{Cf.} \S\ref{sec:wa}) generated from the compound atoms in $\mathcal{M}$: the finite sequence $t_\sigma$ generated from the log trace $\sigma$ is accepted by the automaton $\mathcal{A}_{\varphi_{\mathcal{M}}}$ iff. $\sigma$ is conformant to the model $\mathcal{M}$ (\S\ref{sec:dadtap}). Next, we code $t_\sigma$ and $\mathcal{A}_{\varphi_{\mathcal{M}}}$ as specific automata (\S\ref{ssec:amfta}) that are exploited by a planner to generate the minimally repaired sequence $\hat{t_\sigma}$ of $t_\sigma$ (\S\ref{ssec:eip}), out of which we generate the minimally repaired trace $\hat{\sigma}$ which is conformant to $\mathcal{M}$ (\S\ref{ssec:trerepair}).


%%the finite sequence $t_\sigma$ generated from $\sigma$ by replacing the events with the satisfied atoms in $\Sigma$ is going to be accepted by the automaton $\mathcal{A}_{}$


%% a sequence acceptation task over an automaton by generating a specific set of atoms $\Sigma$ out of \dots

%%In this section, we demonstrate that assessing the conformance checking of a log trace $\sigma$ containing payloads against a (data-aware) Declare model $\mathcal{M}$ can be reduced to the sequence acceptation problem in \S\ref{sec:wa} by choosing a specific set of atoms $\Sigma$ partitioning the data space (\S\ref{sec:dadtap}). Then, we can exploit $\Sigma$ to generate a string sequence $t_\sigma$ from $\sigma$ and a DFA for $\mathcal{M}$, from which we will generate two further automata accepting \cite{XuLZ17a,MaggiMCA18} (\S\ref{ssec:amfta}). Last, we can encode such technique as a planning problem in PDDL by adding a novel \texttt{replacement} planning action (\S\ref{ssec:eip}).

%Given a Declare model $\mathcal{M}=\Set{c_i}_{1\leq i\leq m}$, we can always express $\mathcal{M}$ as one single LTL$_f$ formula $\varphi=\bigwedge_{1\leq i\leq m}\varphi_i$, where $\varphi_i$ is the LTL$_f$ translation of the Declare constraint $c_i\in\mathcal{M}$ \cite{LeoniMA12}.
%
%
%
%We can then translate  \cite{0016921} and that LTL$_f$ formulae can be modeled as deterministic finite-state automata (DFAs) \cite{Westergaard11,Lydia}.
%
%Given that Declare semantics can be expressed as LTL$_f$, we can directly analyse such language having the following syntax:
%\[\varphi::=\phi\;|\;\neg \varphi\;|\;\varphi_1\wedge\varphi_2\;|\;\Next \varphi_1\;|\;\varphi_1\Until\varphi_2\]
%where $\phi\in \mathsf{Prop}$, while $\Next$ and $\Until$ are respectively the \textit{next} and \textit{until} operators. This is the functionally complete set of connectives, with which we can express  disjunction ($\vee$),  logical implication ($\Rightarrow$),  equivalence ($\Leftrightarrow$), globally ($\Globally$), finally ($\Finally$), weak until ($\Wntil$), and release ($\Release$) as in \cite{XuLZ17a}. Given a finite trace $\sigma=\sigma_1\cdots \sigma_n$ of length $|\sigma|=n$, the satisfiability of $\varphi$ over the $i$-the event in $1\leq i\leq |\sigma|$, namely $\sigma_i\vDash \varphi$, is inductively defined as follows:
%\begin{itemize}
%	\item $\sigma_i\vDash\phi$ iff. $\sigma_i\vDash\phi$, $\phi\in \mathsf{Prop}$
%	\item $\sigma_i\vDash\neg\varphi$ iff. $\sigma_i\not\vDash\varphi$
%	\item $\sigma_i\vDash\varphi_1\wedge\varphi_2$ iff. jointly $\sigma_i\vDash\varphi_1$ and $\sigma_i\vDash\varphi_2$
%	\item $\sigma_i\vDash\Next\varphi$ iff. $\sigma_{i+1}\vDash\varphi$ with $1\leq i< |\sigma|$
%	\item $\sigma_i\vDash\varphi_1\Until\varphi_2$ iff. it exists $i\leq j\leq |\sigma|$ such that $\sigma_j\vDash\varphi_2$ and, for each $i\leq k<j$, $\sigma_k\vDash\varphi_1$
%\end{itemize}
%Given the working assumptions in \S\ref{sec:wa} and the interpretation of the Declare templates in LTL$_f$, we can restrict $\phi$ to a propositional formulas containing either the universal truth or falsehoods, or predicates $\psi(\sigma_i)$ in the form $\texttt{A}(\sigma_i)\wedge \phi^d(\sigma_i)$.  We say that $\sigma$ satisfies $\varphi$, namely $\sigma\vDash\varphi$, if $\sigma_1\vDash\varphi$. Given that any  Declare clause can be expressed in terms of LTL$_f$ \cite{10.1007/978-3-642-40176-3_8}, any possible  Declare model can be expressed as the conjunction of the LTL$_f$ representations of the  Declare clauses within the model.
%
%\cite{XuLZ17a} showed that we can reduce the conformance checking strategy into a trace alignment problem by following those subsequent steps.
%
%Firstly,  $\sigma\vDash\varphi$ can be proved by
% \begin{enumerate*}[label=\emph{\alph*})]
%\item  picking an alphabet $\Sigma$,
%\item  picking a transformation $\tau$ of traces $\sigma$ into strings $\tau(\sigma_1\cdots \sigma_n)=\tau(\sigma_1)\cdots \tau(\sigma_n)=t_1\cdots t_n$ in $\Sigma^*$,
%\item  picking a bijection $p_i\xleftrightarrow{f}\psi_i$ between $p_i\in\Sigma$ and atoms $\psi_i$ in $\varphi$, and
%\item  transforming $\varphi$ into a  DFA\footnote{}  $\mathcal{A}$
%\end{enumerate*} %and a bijection $p_i\xleftrightarrow{f}\psi_i$ and $\varphi$ into a finite-state automaton
%such that $t_1\cdots t_n$ is accepted by $\mathcal{A}$ iff. %$\sigma\vDash\varphi$. This happens when
%, for $1\leq i\leq |\sigma|$, it always exists a transition $q_i\xrightarrow{p_i}q_{i+1}$ in $\mathcal{A}$ for which $\psi_i(\sigma_i)$ and, for $i=|\sigma|$, $q_{|\sigma|+1}$ is also an accepting state for $\mathcal{A}$. In the non data-aware  scenario where the set of all the possible activity labels $\textsf{Act}$ is finite and atoms $\psi_i$ are always in the form of $\texttt{A}$ with $\texttt{A}\in\textsf{Act}$ \cite{XuLZ17a,Westergaard11}, this reduces to choose
% \begin{enumerate*}[label=\emph{\alph*})]
%	\item  $\textsf{Act}$ as $\Sigma$,
%	\item  $\lambda$ as the transformation $\tau$,
%	\item  use the immediate bijection $\texttt{A}\xleftrightarrow{f}\texttt{A}$, and
%	\item  to generate a DFA from the automata in Figure~\ref{fig:g1g2} from $\varphi$ by replacing an edge $q_i\xrightarrow{S}q_j$ with $q_i\xrightarrow{\texttt{A}}q_j$ for each $\texttt{A}\in S$.
%\end{enumerate*} On the other hand, we can observe that the representation of $\psi_i$ as a combination of atoms is less straightforward in a general data-aware Declare scenario, as we must guarantee that each event $\sigma_i$ is transformed into one single symbol in $\Sigma$ (see also \S\ref{sec:wa}). For this reason, we firstly show how we can build such set of atoms $\Sigma$:
%

%\begin{lemma}
%It always possible to decompose a proposition $\psi(\sigma_i)=\texttt{A}(\sigma_i)\wedge \phi^d(\sigma_i)$ from an LTL$_f$ interpretation of a Declare model $\mathcal{M}$ via a disjunction of atoms in $\Sigma$ such that each event $\sigma_i$ satisfies only one atom in $\Sigma$.
%\end{lemma}
%\begin{proof}
%Figure~\ref{fig:twoexamples} provides an intuitive sketch of the proof. In more detail, after representing each constraint $c_i\in\mathcal{M}$ (step 1) as an LTL$_f$ formula (step 2) in \textit{negated normal form} (\textit{nnf}), we colle
%\end{proof}
%
%%Secondly, we can transform the constraint automaton $\mathcal{A}$ into an augmented automaton $\mathcal{A}$ accepting all the traces satisfying $\varphi$
%%%
%%\begin{proof}	
%%Given that non-deterministic finite-state automata (NFA) accept the same language of deterministic finite-state automata \cite{0016921} and that LTL$_f$ formulae can be modeled as DFAs \cite{Westergaard11,Lydia}, there exists a transformation for  $\varphi$ into a deterministic finite-state automaton $\mathcal{A}$ (\textit{constraint automaton}) and one for $\sigma$ into a deterministic finite-state automaton $\mathcal{T}$ (\textit{trace automaton}) as a single (accepting) path such that, if $\mathcal{T}$ is (also) an accepting path for $\mathcal{A}$, then $\sigma\vDash\varphi$ \cite{XuLZ17a}.
%%\end{proof}
%%
%%
%%
%%We can show that the declarative conformance checking can be modeled as a trace alignment problem by firstly transforming a Declare model as a deterministic finite-state automaton (DFA) via an LTL$_f$ formula, namely \textit{constraint automaton} $\mathcal{A}$, as well as directly translating the trace $\sigma$ as another DFA, namely \textit{trace automaton} $\mathcal{T}$, which is formed by one single path accepting $\sigma$. Last, we need to show that $\mathcal{A}$ contains the exact path $\mathcal{T}$
%%
%%We can represent $\sigma$ as a DFA $\mathcal{T}=(\Sigma_\sigma,Q_\sigma,q_0^\sigma,\rho_\sigma,F_\sigma)$, namely a \textit{trace automaton} \cite{XuLZ17a}, having having \begin{enumerate*}[label=\emph{\alph*})]
%%	\item $\Sigma_\sigma=\Set{\sigma_1,\dots,\sigma_n}$,
%%	\item $Q_\sigma=\Set{q_0^\sigma,\dots,q_n^\sigma}$ a set of arbitrary $|\sigma|+1$ states, with
%%	\item an initial state $q_0^\sigma$ and
%%	\item a set $F=\Set{q_n^\sigma}$ of accepting states, where
%%	\item the transition relation $\rho_\sigma(q_i^\sigma,\sigma_i)=q_{i+1}^\sigma$ for each $1\leq i\leq |\sigma|+1$.
%%\end{enumerate*}
%%Every LTL$_f$ formula can be directly associated to a deterministic finite-state automaton (DFA) $\mathcal{A}=(\Sigma,Q,q_0,\rho,F)$, namely a \textit{constraint automaton}, accepting only the traces satisfying $\varphi$ \cite{Lydia}, having \begin{enumerate*}[label=\emph{\alph*})]
%%	\item an input alphabet $\Sigma\subseteq \textsf{Prop}$,
%%	\item a finite set $Q$ of states, with
%%	\item an initial state $q_0\in Q$ and
%%	\item a set $F\subseteq Q$ of accepting states, where
%%	\item the latter can be reached from the former via a transition relation $\rho\colon Q\times \Sigma\to Q$.
%%\end{enumerate*} Furthermore,
%%
%%\[s_{\mathcal{A},\sigma}(q,i)=\begin{cases}
%% 	\textbf{true} & i>|\sigma|\wedge q\in F\\
%% 	s_{\mathcal{A},\sigma}(q,i+1) & i\leq |\sigma|\wedge \exists! \phi. \rho(q,\phi)
%%\end{cases}\]
%%
%%\bigskip
%
%Secondly, the authors introduced \textit{repair atoms}  \texttt{del\_a} (or \texttt{add\_a}) for each $\texttt{A}\in\Sigma$ respectively remarking that \texttt{A} was removed (or added) in the input trace, while it needs to be added (removed) to make the string accepted. This allows to transform $\mathcal{A}$ into $\mathcal{A}^+$ accepting all the repaired strings $\tilde{t}$ from $t$ \texttt{\color{red}[TODO]}
%
%





\subsection{$\Sigma$-encoding for conformance checking}\label{sec:dadtap}
As per previous considerations, we want to show that it is sufficient to provide a specific characterization of $\Sigma$, which will be used to generate an automaton accepting symbols in $\Sigma$ and transforming traces as finite sequences in $\Sigma^*$. The proposed approach for obtaining $\Sigma$ from the Declare model $\varphi_{\mathcal{M}}$ is sketched in Figure~\ref{fig:twoexamples}, and described in detail in the following steps.

\begin{figure}[!t]
	%{\hspace{-1.3cm}\includegraphics[width=1.3\textwidth]{images/example_1}}
	{\hspace{-1.3cm}\includegraphics[width=1.3\textwidth]{images/example_3}}
	\caption{Intermediate steps required for obtaining $\Sigma=\Set{p_i|1\leq i\leq 9}\cup\{\texttt{A}\}$ from $\mathcal{M}$ and transforming $\mathcal{L}=\Set{\sigma,\sigma'}$ to a set of finite sequences $T=\Set{t_\sigma,t_{\sigma'}}$, as well as replacing atoms in $\varphi_{\mathcal{M}}$ with equivalent atoms in $\Sigma$ ($\varphi_{\mathcal{M}}'$).}\label{fig:twoexamples}
\end{figure}
In step 1, we exploit the usual conversion of each single Declare clause into a LTL$_f$ formula in the \textit{negated normal form} \cite{LiPZVR20}, where negation is possibly pushed inside atoms ``$\texttt{a}.k\;\Re\; c$'' by replacing $\Re$ with its negation.

In step 2, for each compound condition $\psi=\phi_{\texttt{a}}\wedge \phi^d$ over labels $\texttt{a}\in\textsf{Act}$, we collect all the atoms in $\phi_d$ in the form ``$\texttt{a}.k\;\Re\; c$'' for $k\in K$ in a map $\mu(\texttt{a},k)$. Contextually, we represent each atom as an interval, and we \textit{decompose} them
%
%%We now describe the main contribution of the paper, namely a technique for computing log trace alignments over Declare data-aware models. Our approach takes as input \begin{enumerate*}[label=\emph{\alph*})]
%	\item a Declare data-aware model $\mathcal{M}$ expressed as a set of instantiated templates,
%	\item a log trace $\sigma$,
%\end{enumerate*} and ranks the outcome of a LTL$_f$ conformance checking $\sigma\tilde{\vDash}\varphi$ accordingly to a data distance function $\mathcal{D}$.
%
%%The input transformation for reducing the data-aware alignment problem to the data-agnostic one is presented in Figure~\ref{fig:twoexamples} for two alignment examples. In particular, we first transform the data-aware Declare model, for then collecting the required information for providing the trace transformation.
%
%%\textbf{data-aware Declare Model Processing.}
%
%
%
%
%%In step 3, we collect the data-aware predicates ``$\texttt{A}.\textit{var}\;\Re\;c$'' from all the model's clauses and group them by $\texttt{A}.\textit{var}$; each of these predicates is \textit{decomposed}
into a disjunction maximal non-overlapping data-aware predicates. This task can be efficiently computed via interval trees \cite{inttree}. E.g., predicates $\texttt{B}.x>3$ and $\texttt{B}.x>0$ are first represented as intervals $\interval({3,+\infty})$ and $\interval({0,+\infty})$, and then decomposed into disjoint sub-invervals $\interval({-\infty,0}]$, $\interval[{0,3}]$, and $\interval({3,+\infty})$. Last, we replace the atoms in each LTL$_f$ formula by its decomposed representation, if any.


In step 3, we put an atom $\texttt{a}\in\textsf{Act}$ in $\Sigma$ if the map $\mu(\texttt{a},k)$ is empty for each key $k\in K$; otherwise, given all the keys $k_{\texttt{a}_1},\dots,k_{\texttt{a}_h}\in K$ for which the map $\mu(\texttt{a},k_{\texttt{a}_i})$ is not empty, we partition the data space by combining the non-overlapping intervals as $\mu(\texttt{a},k_{\texttt{a}_1})\times\cdots\times\mu(\texttt{a},k_{\texttt{a}_h})$ obtained from the previous step. For each of this interval combination, we generate a fresh atom and put it in $\Sigma$. E.g., label \texttt{A} is never associated to a data condition, and therefore it will be associated to one single atom \texttt{A}. Concerning the label \texttt{B}, it is associated to data intervals over keys $x$ and $y$, which induce a space partitioning of 9 intervals, for which we generate distinct atoms $p_1\dots p_9$. As a result, we obtain $\Sigma=\Set{p_i|1\leq i\leq 9}\cup\Set{\texttt{A}}$.


%In step 4, for each event label \texttt{A}, we partition the data space \textit{var}$_1\times\dots\times$\textit{var}$_h$ associated to \texttt{A} by exploiting the disjoint intervals mined in the previous step. Each of such combination will be syntactically represented as a fresh \textit{atom}  proposition $p_i$: this implies that the label \texttt{A} is represented by the disjunction $\bigvee_ip_i$. When the data space associated to the predicates mined for \texttt{A} has only one property,  the atoms corresponds to the ones mined in the previous step. E.g., \texttt{A} in the first example from \ref{fig:twoexamples} is equivalent to $p_1\vee p_2\vee p_3$, and $\texttt{A}.\textit{x}<5$ is rewritten as $p_1$; given that such atoms represent disjoint intervals, then $\texttt{A}\wedge p_1\equiv p_1$. Similarly, $\neg \texttt{A}\vee \texttt{A}.\textit{x}<5\vee 5\leq\texttt{A}.\textit{x}\leq 10$ can be immediately rewritten as $\neg(p_1\vee p_2\vee p_3)\vee p_1\vee p_2$, which is equivalent to $\neg p_3\vee p_1\vee p_2$. Last, each LTL$_f$ representation of a data-aware Declare clause is represented into one single LTL$_f$ formula by conjunction and simplification.

Given the previously generated atoms, we can now generate a finite sequence $t_\sigma\in T$ for each log trace $\sigma\in\mathcal{L}$, and replace the compound conditions in the LTL$_f$ interpretation $\varphi_{\mathcal{M}}$ of model $\mathcal{M}$ with a disjunction of atoms from $\Sigma$ as described in \S\ref{sec:wa}, thus obtaining an equivalent LTL$_f$ formula $\varphi_{\mathcal{M}}'$.
%
%\textbf{data-aware Log Trace Processing.} Given the atomization in Step 4, we process each data-enriched event within the trace as follows: if the event label is never associated with a data predicate, then we just discard the data information; otherwise, we replace each event with the single corresponding atom satisfying the associated semantics. Please observe that, by previous construction, each event can be represented by just one possible propositional atom, as the previous construction guarantees a partitioning (thus non-overlapping) representation of the data space.
E.g., all the trace events from Figure~\ref{fig:twoexamples} labelled as \texttt{A} are replaced with the atom \texttt{A}, as there are no (data) conditions in the model $\mathcal{M}$ that we can exploit to partition the data space. On the other hand, each data condition \texttt{B} is replaced by an equivalent atom in $\Sigma$: event \texttt{B\{x=1,y=0\}} is uniquely represented by $p_5$, while event \texttt{B\{x=10,y=0\}} is uniquely represented by atom $p_8$. Similar considerations can be drawed for the $\psi$ compound atoms in $\varphi_{\mathcal{M}}$ where compound conditions are replaced into an equivalent disjunction of atoms in $\Sigma$: $\texttt{B}.x>0$ will be described by all the possible configurations of $y$ and data intervals $0<\texttt{B}.x\leq 3$ and $\texttt{B}.x>3$, which are identified by the disjunction $p_4\vee p_5\vee p_6\vee p_7\vee p_8\vee p_9$. On the other hand, $\texttt{B}.x>3\wedge \texttt{B}.y=0$ can be directly mapped to atom $p_8$: this results into generating an equivalent formula $\varphi_{\mathcal{M}}'$ out of $\varphi_{\mathcal{M}}$.

After generating $\varphi_{\mathcal{M}}'$, we can exploit directly existing approaches \cite{LeoniMA12,Westergaard11,Lydia} to generate DFAs, thus obtaining Figure \ref{fig:g1g2}: the first trace will be never accepted by the model, as well as the first sequence is never accepted by the associated automaton. Similarly, the second trace is accepted by the model, as well as the second sequence is accepted by the DFA. In the forthcoming subsection, we will discuss how to generate repaired sequences that are accepted by the model.

\subsection{Automaton Manipulation for Trace Alignment}\label{ssec:amfta}
Let us now consider a string sequence $t_\sigma=t_1\cdots t_n$ generated from a log trace $\sigma$ in the previous section, and the constraint automaton $\mathcal{A}_{\varphi_{\mathcal{M}}}$ generated from the Declare model $\mathcal{M}$, both generated via a set of atoms $\Sigma$. If the trace was deviant with respect to the model, we are interested in generating a repair sequence $\varrho=\varrho_1\cdots \varrho_m$ from $t_\sigma$ describing the operations to perform over $\sigma$ to make it conformant to the model $\mathcal{M}$.

To realize this transformation, we consider two kinds of atomic violations, which can be caused by wrong (\textit{deletion}) or missing (\textit{insertion}) activities with data conditions in $\Sigma$. Differently from the non-data aware contexts such as \cite{XuLZ17a,MaggiMCA18}, we also need to model \textit{replacement} operations, defined as data updates within one single trace event: these can be mimicked by delete operations followed by insertion ones, as they substitute an event within a trace violating the model  with a conforming one. Such operations can be defined as follows:
\begin{itemize}
	%\item synchronization $[\sigma_k\leftrightarrow \phi]$ aborts if $\sigma_k\neq\phi$, for  $1\leq k\leq |\sigma|$
	\item \textit{deletion}\,\, $[\#\sigma_k\leftarrow \phi]::= \sigma_1\cdots\sigma_{k-1}\sigma_{k+1}\cdots \sigma_n$,\,\,\, for $n=|\sigma|$, $1\leq k\leq n$, and $\phi=\sigma_k$
	\item \textit{insertion} $[@\sigma_k\leftarrow \phi]::= \sigma_1\cdots\sigma_{k-1}\phi\sigma_{k}\cdots \sigma_n$,\,\,\,\,\,\,\,\,\,\,\,\,\, for $n=|\sigma|$ and $1\leq k\leq n$
	\item \textit{replacement} $[\sigma_k[\phi\mapsto\phi']]::=\sigma_1\cdots \sigma_{k-1}\phi'\sigma_{k+1}\cdots\sigma_n$ for $n=|\sigma|$, $1\leq k\leq n$, and $\phi=\sigma_k$
\end{itemize}
Each of these operations has an associated cost, either quantifying the severity of the found violation or determining which operations shall be preferred. E.g., by assigning a higher cost to insertions and deletions and a lower one to replacements, we will favor replacements when possible. The \textit{transformation cost} is defined as the number of deletions multiplied by their cost, plus the number of insertions multiplied by their cost, plus the number of replacements multiplied by their cost.

We can now define the conformance checking problem as follows:
\begin{definition}[Log/Declare Conformance Checking]
Given a trace $\sigma$ and a Declare model $\mathcal{M}$, either $\sigma$ conforms to $\mathcal{M}$, or $\sigma$ is deviant and it exists a repair sequence $\varrho$ both making $\sigma$ non-deviant for $\mathcal{M}$ and guaranteeing a minimal transformation cost.
\end{definition}

%\texttt{\color{red}[TODO]} we consider insertions and deletions as possible repairs, while substitutions can be modeled by deletions followed by insertions. Synchronizations are \texttt{noops} requiring that a trace $\sigma$ at step $k$ contains a predicate $\phi$.
%%
%Therefore, any repair  of a trace $\sigma$ can be expressed in terms of a sequence of operations $\texttt{op}_1\cdots \texttt{op}_m$ which, when executed in appearance order, generate a novel trace $\tilde{\sigma}$ from $\sigma$.  \texttt{\color{red}[TODO]}
%%
%
%Last, the amount of repairs can be numerically quantified using a cost function $\mathcal{C}$ returning zero for any synchronization and $1$ otherwise; therefore $cost(\sigma, \tilde{\sigma})$ returns the minimal number of non-synchronization operations\footnote{Formally, $cost(\sigma,\tilde{\sigma})=\min_{\substack{\texttt{op}_1\cdots \texttt{op}_m,\\(\texttt{op}_m\,\circ \cdots\circ\, \texttt{op}_1)(\sigma)=\tilde{\sigma}}}\sum_{1\leq i\leq m}\mathcal{C}(\texttt{op}_m)$} required to obtain $\tilde{\sigma}$ from $\sigma$. Therefore, the conformance checking of a log trace $\sigma$ against a  Declare model represented as an LTL$_f$ formula $\varphi$ as in \cite{XuLZ17a} either returns $\sigma$ with cost zero if $\varphi\vDash\varsigma$ or, otherwise, returns a set of pairs $\Set{\braket{\tilde{\sigma},\texttt{op}_1\cdots\texttt{op}_m}_i}_{1\leq i\leq k, k\in\mathbb{N}}$, where\footnote{Formally, $\sigma\tilde{\vDash}\varphi = \Set{\braket{\tilde{\sigma},\texttt{op}_1\cdots\texttt{op}_m} | cost(\sigma,\tilde{\sigma}) = \min_\mu cost(\sigma,\mu),\;\tilde{\sigma}\vDash\varphi,\; (\texttt{op}_m\,\circ \cdots\circ\, \texttt{op}_1)(\sigma)=\tilde{\sigma}}$.} each trace $\tilde{\sigma}\in S$ is conformant to $\varphi$ and minimizes the alignment cost $cost(\sigma,\tilde{\sigma})$ via a repair sequence $\texttt{op}_1\cdots\texttt{op}_m$. We denote the output of such conformance checking as $\sigma\tilde{\vDash}\varphi$.

The process of generating a repair sequence can be addressed by resorting to DFAs (\S\ref{sec:wa}). Let $t_\sigma=t_1\cdots t_n$ be a string sequence generated from a log trace $\sigma$ via $\Sigma$, $\mathcal{A}_{\varphi_{\mathcal{M}}}=(\Sigma,Q,q_0,\rho,F)$ the constraint automaton to check $t_\sigma$ against. From $t_\sigma$, we define a further automaton, called the \textit{path automaton} $\mathcal{T}=(\Sigma_t,Q_t,q_0^t,\rho_t,F_t)$ having \begin{enumerate*}[label=\emph{\alph*})]
	\item $\Sigma_t=\Set{t_i|t_i\in t_\sigma}$,
	\item $Q_t=\Set{q_0^t,\cdots,q_n^t}$ as a set of $|t_\sigma|+1$ states,
	\item $\rho(q_i^t,e_{i+1})=q_{i+1}^t$ for $0\leq i\leq n-1$,
	and
	\item $F_t={q_n^t}$.
\end{enumerate*} By definition, such graph accepts only $t_\sigma$.

\begin{figure}[!t]
	\centering
	{\includegraphics[width=.7\textwidth]{images/Tplus}}
	\caption{Augmented path automaton $\mathcal{T}^+$ for $t_{\sigma'}=p_5\;\texttt{A}\;\texttt{A}$.}\label{fig:tplus}
\end{figure} \begin{figure}[!t]
\centering
{\includegraphics[width=.7\textwidth]{images/Aplus}}
\caption{Augmented path automaton $\mathcal{A}_{\varphi_{\tiny\mathcal{M}}}^+$ for $\mathcal{A}_{\varphi_{\tiny\mathcal{M}}}$.}\label{fig:aplus}
\end{figure}
Next, we augment $\mathcal{T}$ and $\mathcal{A}_{\varphi_{\mathcal{M}}}$ by adding transitions related to just the atomic operations of insertions and deletions: Thus, from $\mathcal{T}$ we generate the automaton $\mathcal{T}^+=(\Sigma_t^+,Q_t,q_0^t,\rho_t^+,F_t)$ having:
\begin{itemize}
	\item $\Sigma_t^+$ extending $\Sigma_t\subseteq \Sigma$ by adding an insertion $\textit{ins\_}\phi$ for each atom $\phi\in\Sigma_t\cup\Sigma$ and a deletion $\textit{del\_}\phi$ for each atom  $\phi\in\Sigma_t$.
	\item $\rho_t^+$ extending $\rho_t$ by adding deletions $\rho_t^+(p,\textit{del\_}\phi)=q$ for each transition $\rho_t(p,\phi)=q$; and, for all atoms $\phi\in\Sigma\cup\Sigma_t$ and states $q\in Q_t$, we add insertions $\rho_t^+(q,\textit{ins\_}\phi)=q$.
\end{itemize}
Figure~\ref{fig:tplus} shows the path automaton generated from the deviant trace $\sigma_1$ from Figure~\ref{fig:twoexamples}. Similarly, from $\mathcal{A}_{\varphi_{\mathcal{M}}}$ we obtain $\mathcal{A}_{\varphi_{\tiny\mathcal{M}}}^+=(\Sigma^+,Q,q_0,\rho^+,F)$ having:
\begin{itemize}
	\item $\Sigma^+$ extending $\Sigma$ by adding an insertion $\textit{ins\_}\phi$ for each atom $\phi\in\Sigma$ and a deletion $\textit{del\_}\phi$ for each atom  $\phi\in\Sigma\cup\Sigma_t$.
\item $\rho^+$ extending $\rho_t$ by adding insertions $\rho^+(p,\textit{ins\_}\phi)=q$ for each transition $\rho(p,\phi)=q$; and, for all atoms $\phi\in\Sigma\cup\Sigma_t$ and states $q\in Q$, we add deletions $\rho_t^+(q,\textit{del\_}\phi)=q$.
\end{itemize}
Figure~\ref{fig:aplus} shows the automaton augmented with the repair operations $\mathcal{A}_{\varphi_{\tiny\mathcal{M}}}^+$ obtained for the Declare model $\mathcal{M}$ in Figure~\ref{fig:twoexamples} via $\Sigma$. Intuitively, $\mathcal{A}_{\varphi_{\tiny\mathcal{M}}}^+$ accepts all the string sequences conformant to the model and have been obtained by adding/removing the missing/wrong atoms to/from $t_\sigma$, where atomic operations are explicitly marked. As required, both augmented automatons will never accept $t_{\sigma'}=p_5\;\texttt{A}\;\texttt{A}$. However, if we repair the sequence by adding $p_8$ at the end and explicitly remarking such repair with $\textit{ins\_}p_8$, then all the augmented automata will accept $\hat{t_{\sigma'}}=p_5\;\texttt{A}\;\texttt{A}\;\textit{ins\_}p_8$.

Next, we show how automated planners searching for the repair operations $\varrho$ that are going to be exploited to repair the trace $\sigma$ via the previously designed automata.

\subsection{Encoding in PDDL}\label{ssec:eip}
\texttt{\color{red}[TODO]}
\\

E.g., the alignment result $\hat{t_\sigma}=p_5\;\texttt{A}\;\texttt{A}\;\textit{ins\_}p_8$ of trace $\sigma=$\texttt{B\{x=1,y=0\}A\{x=6\}\\A\{x=4\}} generates the repair $\varrho=[@\sigma_4\leftarrow p_8]$ after removing the \texttt{sync} operations.

\subsection{Trace repair}\label{ssec:trerepair}
Last, we need to leverage the repair actions generated by the planner to repair the data trace. In particular, the generated repair actions are always ordered and incrementally change different positions within the trace. By removing all the \texttt{sync}s from the planner, we will obtain a sequence of insertion $[@\sigma_k\leftarrow \phi]$, deletion $[\#\sigma_k\leftarrow \phi]$, and replacement $[\sigma_k[\phi\mapsto \phi']]$ operations for a trace $\sigma$ via is associated $t_\sigma$. While deletions $[\#\sigma_k\leftarrow \phi]$ can be trivially implemented in the data-aware scenario by simply removing the problematical event at position $k$ from $\sigma$, for insertions and replacements we need to add events with their associated payloads or adapt such values. Replacements $[\sigma_k[\phi\mapsto \phi']]$ can be easily be implemented by replacing the values in $\sigma_k$ violating data conditions $\phi'$ expressed with a conjunction of data intervals over different keys with the nearest values in $\phi'$ to the values in $\sigma_k$. On the other hand, insertions require to generate totally new values: given that traces $\sigma=\sigma_1\dots \sigma_n$ model temporal activities starting at $\sigma_1$ and ending at $\sigma_n$,  the insertion $[@\sigma_k\leftarrow \phi]$ of a new event compliant with $\phi$ at position $k$ can be modeled by generating a new event having the trace label \texttt{a} induced by $\phi$, which is then instantiated with the same data values present in the last occurrence of an event similarly labeled (i.e., \texttt{a}) if any, and instantiated with default values otherwise; next, such values are repaired by choosing the values in $\phi$ nearer to the one in the newly added event.

E.g., given the repair $\varrho=[@\sigma_4\leftarrow p_8]$ generated for a trace $\sigma=$\texttt{B\{x=1,y=0\}\\A\{x=6\}A\{x=4\}}, we obtain a new trace $\sigma=$\texttt{B\{x=1,y=0\}$  $A\{x=6\}A\{x=4\}B\{x=4,y=0\}}, where \texttt{4} is the nearest integer to \texttt{B.x=1} from the first event which is compliant to $p_8\equiv\texttt{B}.x>3\wedge \texttt{B}.y=0$. 
\section{Experiments}
\label{sec:experiments}

\pgfplotstableread[col sep=comma,header=true]{experiments/3-CONSTRAINTS-0-NOISE-ms.csv}\datafileANOISE
\pgfplotstablecreatecol[copy column from table={\datafileANOISE}{[index] 0}] {length30} {\datafileANOISE}
\pgfplotstablecreatecol[copy column from table={\datafileANOISE}{[index] 1}] {symba30} {\datafileANOISE}
\pgfplotstablecreatecol[copy column from table={\datafileANOISE}{[index] 2}] {actions30} {\datafileANOISE}

\pgfplotstableread[col sep=comma,header=true]{experiments/3-CONSTRAINTS-1-NOISE-ms.csv}\datafileBNOISE
\pgfplotstablecreatecol[copy column from table={\datafileBNOISE}{[index] 0}] {length31} {\datafileANOISE}
\pgfplotstablecreatecol[copy column from table={\datafileBNOISE}{[index] 1}] {symba31} {\datafileANOISE}
\pgfplotstablecreatecol[copy column from table={\datafileBNOISE}{[index] 2}] {actions31} {\datafileANOISE}

\pgfplotstableread[col sep=comma,header=true]{experiments/3-CONSTRAINTS-2-NOISE-ms.csv}\datafileCNOISE
\pgfplotstablecreatecol[copy column from table={\datafileCNOISE}{[index] 0}] {length32} {\datafileANOISE}
\pgfplotstablecreatecol[copy column from table={\datafileCNOISE}{[index] 1}] {symba32} {\datafileANOISE}
\pgfplotstablecreatecol[copy column from table={\datafileCNOISE}{[index] 2}] {actions32} {\datafileANOISE}

\pgfplotstableread[col sep=comma,header=true]{experiments/3-CONSTRAINTS-3-NOISE-ms.csv}\datafileDNOISE
\pgfplotstablecreatecol[copy column from table={\datafileDNOISE}{[index] 0}] {length33} {\datafileANOISE}
\pgfplotstablecreatecol[copy column from table={\datafileDNOISE}{[index] 1}] {symba33} {\datafileANOISE}
\pgfplotstablecreatecol[copy column from table={\datafileDNOISE}{[index] 2}] {actions33} {\datafileANOISE}

\pgfplotstableread[col sep=comma,header=true]{experiments/5-CONSTRAINTS-0-NOISE-ms.csv}\datafileENOISE
\pgfplotstablecreatecol[copy column from table={\datafileENOISE}{[index] 0}] {length50} {\datafileANOISE}
\pgfplotstablecreatecol[copy column from table={\datafileENOISE}{[index] 1}] {symba50} {\datafileANOISE}
\pgfplotstablecreatecol[copy column from table={\datafileENOISE}{[index] 2}] {actions50} {\datafileANOISE}

\pgfplotstableread[col sep=comma,header=true]{experiments/5-CONSTRAINTS-1-NOISE-ms.csv}\datafileFNOISE
\pgfplotstablecreatecol[copy column from table={\datafileFNOISE}{[index] 0}] {length51} {\datafileANOISE}
\pgfplotstablecreatecol[copy column from table={\datafileFNOISE}{[index] 1}] {symba51} {\datafileANOISE}
\pgfplotstablecreatecol[copy column from table={\datafileFNOISE}{[index] 2}] {actions51} {\datafileANOISE}

\pgfplotstableread[col sep=comma,header=true]{experiments/5-CONSTRAINTS-2-NOISE-ms.csv}\datafileGNOISE
\pgfplotstablecreatecol[copy column from table={\datafileGNOISE}{[index] 0}] {length52} {\datafileANOISE}
\pgfplotstablecreatecol[copy column from table={\datafileGNOISE}{[index] 1}] {symba52} {\datafileANOISE}
\pgfplotstablecreatecol[copy column from table={\datafileGNOISE}{[index] 2}] {actions52} {\datafileANOISE}

\pgfplotstableread[col sep=comma,header=true]{experiments/5-CONSTRAINTS-3-NOISE-ms.csv}\datafileHNOISE
\pgfplotstablecreatecol[copy column from table={\datafileHNOISE}{[index] 0}] {length53} {\datafileANOISE}
\pgfplotstablecreatecol[copy column from table={\datafileHNOISE}{[index] 1}] {symba53} {\datafileANOISE}
\pgfplotstablecreatecol[copy column from table={\datafileHNOISE}{[index] 2}] {actions53} {\datafileANOISE}

\pgfplotstableread[col sep=comma,header=true]{experiments/7-CONSTRAINTS-0-NOISE-ms.csv}\datafileENOISE
\pgfplotstablecreatecol[copy column from table={\datafileENOISE}{[index] 0}] {length70} {\datafileANOISE}
\pgfplotstablecreatecol[copy column from table={\datafileENOISE}{[index] 1}] {symba70} {\datafileANOISE}
\pgfplotstablecreatecol[copy column from table={\datafileENOISE}{[index] 2}] {actions70} {\datafileANOISE}

\pgfplotstableread[col sep=comma,header=true]{experiments/7-CONSTRAINTS-1-NOISE-ms.csv}\datafileFNOISE
\pgfplotstablecreatecol[copy column from table={\datafileFNOISE}{[index] 0}] {length71} {\datafileANOISE}
\pgfplotstablecreatecol[copy column from table={\datafileFNOISE}{[index] 1}] {symba71} {\datafileANOISE}
\pgfplotstablecreatecol[copy column from table={\datafileFNOISE}{[index] 2}] {actions71} {\datafileANOISE}

\pgfplotstableread[col sep=comma,header=true]{experiments/7-CONSTRAINTS-2-NOISE-ms.csv}\datafileGNOISE
\pgfplotstablecreatecol[copy column from table={\datafileGNOISE}{[index] 0}] {length72} {\datafileANOISE}
\pgfplotstablecreatecol[copy column from table={\datafileGNOISE}{[index] 1}] {symba72} {\datafileANOISE}
\pgfplotstablecreatecol[copy column from table={\datafileGNOISE}{[index] 2}] {actions72} {\datafileANOISE}

\pgfplotstableread[col sep=comma,header=true]{experiments/7-CONSTRAINTS-3-NOISE-ms.csv}\datafileHNOISE
\pgfplotstablecreatecol[copy column from table={\datafileHNOISE}{[index] 0}] {length73} {\datafileANOISE}
\pgfplotstablecreatecol[copy column from table={\datafileHNOISE}{[index] 1}] {symba73} {\datafileANOISE}
\pgfplotstablecreatecol[copy column from table={\datafileHNOISE}{[index] 2}] {actions73} {\datafileANOISE}

\pgfplotstableread[col sep=comma,header=true]{experiments/10-CONSTRAINTS-0-NOISE-ms.csv}\datafileINOISE
\pgfplotstablecreatecol[copy column from table={\datafileINOISE}{[index] 0}] {length90} {\datafileANOISE}
\pgfplotstablecreatecol[copy column from table={\datafileINOISE}{[index] 1}] {symba90} {\datafileANOISE}
\pgfplotstablecreatecol[copy column from table={\datafileINOISE}{[index] 2}] {actions90} {\datafileANOISE}

\pgfplotstableread[col sep=comma,header=true]{experiments/10-CONSTRAINTS-1-NOISE-ms.csv}\datafileLNOISE
\pgfplotstablecreatecol[copy column from table={\datafileLNOISE}{[index] 0}] {length91} {\datafileANOISE}
\pgfplotstablecreatecol[copy column from table={\datafileLNOISE}{[index] 1}] {symba91} {\datafileANOISE}
\pgfplotstablecreatecol[copy column from table={\datafileLNOISE}{[index] 2}] {actions91} {\datafileANOISE}

\pgfplotstableread[col sep=comma,header=true]{experiments/10-CONSTRAINTS-2-NOISE-ms.csv}\datafileMNOISE
\pgfplotstablecreatecol[copy column from table={\datafileMNOISE}{[index] 0}] {length92} {\datafileANOISE}
\pgfplotstablecreatecol[copy column from table={\datafileMNOISE}{[index] 1}] {symba92} {\datafileANOISE}
\pgfplotstablecreatecol[copy column from table={\datafileMNOISE}{[index] 2}] {actions92} {\datafileANOISE}

\pgfplotstableread[col sep=comma,header=true]{experiments/10-CONSTRAINTS-3-NOISE-ms.csv}\datafileNNOISE
\pgfplotstablecreatecol[copy column from table={\datafileNNOISE}{[index] 0}] {length93} {\datafileANOISE}
\pgfplotstablecreatecol[copy column from table={\datafileNNOISE}{[index] 1}] {symba93} {\datafileANOISE}
\pgfplotstablecreatecol[copy column from table={\datafileNNOISE}{[index] 2}] {actions93} {\datafileANOISE}

\begin{table*}[t!]
\begin{center}
\resizebox{\textwidth}{!}{\newcolumntype{C}{>{\centering\arraybackslash}p{18mm}}
\pgfplotstabletypeset[
	%every even row/.style={
	%	before row={\rowcolor[gray]{0.9}}
	%},
font=\scriptsize,
	every head row/.style={
    output empty row,
% 		before row=\toprule,
		%after row=\midrule,
		before row={%
              \toprule
              Trace length
            & Alignment Time
            & Alignment Cost
            & Alignment Time
            & Alignment Cost
            & Alignment Time
            & Alignment Cost
            & Alignment Time
            & Alignment Cost
            \\
		},
		after row={%
            \midrule
              \multicolumn{1}{c|}{\textbf{\emph{0 const. modified}}}
            & \multicolumn{2}{p{20mm}|}{3 constraints}
			& \multicolumn{2}{p{20mm}|}{5 constraints}
            & \multicolumn{2}{p{20mm}|}{7 constraints}
            & \multicolumn{2}{p{20mm}|}{10 constraints}
            \\\midrule
		},
	},
	%
	columns={length30,symba30,actions30,symba50,actions50,symba70,actions70,symba30,actions30},		
    columns/{length30}/.style={column name={},column type=C},
    columns/{symba30}/.style={column name={},column type=C},
    columns/{actions30}/.style={column name={},column type=C},
    columns/{symba50}/.style={column name={},column type=C},
    columns/{actions50}/.style={column name={},column type=C},
    columns/{symba70}/.style={column name={},column type=C},
    columns/{actions70}/.style={column name={},column type=C},
    columns/{symba30}/.style={column name={},column type=C},
    columns/{actions30}/.style={column name={},column type=C},
	%%
    every row 0 column 5/.style={postproc cell content/.style={@cell content={-}}},
    every row 0 column 6/.style={postproc cell content/.style={@cell content={-}}},
    every row 0 column 7/.style={postproc cell content/.style={@cell content={-}}},
    every row 0 column 8/.style={postproc cell content/.style={@cell content={-}}},
    every row 1 column 7/.style={postproc cell content/.style={@cell content={-}}},
    every row 1 column 8/.style={postproc cell content/.style={@cell content={-}}},
    %every row 0 column 0/.style={postproc cell content/.style={@cell content={1-50}}},
    %every row 1 column 0/.style={postproc cell content/.style={@cell content={51-100}}},
    %every row 2 column 0/.style={postproc cell content/.style={@cell content={101-150}}},
    %every row 3 column 0/.style={postproc cell content/.style={@cell content={151-200}}},
]{\datafileANOISE}
}
\resizebox{\textwidth}{!}{\newcolumntype{C}{>{\centering\arraybackslash}p{18mm}}
\pgfplotstabletypeset[
	%every even row/.style={
	%	before row={\rowcolor[gray]{0.9}}
	%},
font=\scriptsize,
	every head row/.style={
    output empty row,
		before row={%
            \midrule
              \multicolumn{1}{c|}{\textbf{\emph{1 const. modified}}}
            & \multicolumn{2}{p{20mm}|}{3 constraints}
			& \multicolumn{2}{p{20mm}|}{5 constraints}
            & \multicolumn{2}{p{20mm}|}{7 constraints}
            & \multicolumn{2}{p{20mm}|}{10 constraints}
            \\\midrule
		},
	},
	%
	columns={length31,symba31,actions31,symba51,actions51,symba71,actions71,symba91,actions91},		
    columns/{length31}/.style={column name={},column type=C},
    columns/{symba31}/.style={column name={},column type=C},
    columns/{actions31}/.style={column name={},column type=C},
    columns/{symba51}/.style={column name={},column type=C},
    columns/{actions51}/.style={column name={},column type=C},
    columns/{symba71}/.style={column name={},column type=C},
    columns/{actions71}/.style={column name={},column type=C},
    columns/{symba91}/.style={column name={},column type=C},
    columns/{actions91}/.style={column name={},column type=C},
	%%
    every row 0 column 5/.style={postproc cell content/.style={@cell content={-}}},
    every row 0 column 6/.style={postproc cell content/.style={@cell content={-}}},
    every row 0 column 7/.style={postproc cell content/.style={@cell content={-}}},
    every row 0 column 8/.style={postproc cell content/.style={@cell content={-}}},
    every row 1 column 7/.style={postproc cell content/.style={@cell content={-}}},
    every row 1 column 8/.style={postproc cell content/.style={@cell content={-}}},
]{\datafileANOISE}
}
\resizebox{\textwidth}{!}{\newcolumntype{C}{>{\centering\arraybackslash}p{18mm}}
\pgfplotstabletypeset[
	%every even row/.style={
	%	before row={\rowcolor[gray]{0.9}}
	%},
font=\scriptsize,
	every head row/.style={
    output empty row,
		before row={%
            \midrule
              \multicolumn{1}{c|}{\textbf{\emph{2 const. modified}}}
            & \multicolumn{2}{p{20mm}|}{3 constraints}
			& \multicolumn{2}{p{20mm}|}{5 constraints}
            & \multicolumn{2}{p{20mm}|}{7 constraints}
            & \multicolumn{2}{p{20mm}|}{10 constraints}
            \\\midrule
		},
	},
	%
	columns={length32,symba32,actions32,symba52,actions52,symba72,actions72,symba92,actions92},		
    columns/{length32}/.style={column name={},column type=C},
    columns/{symba32}/.style={column name={},column type=C},
    columns/{actions32}/.style={column name={},column type=C},
    columns/{symba52}/.style={column name={},column type=C},
    columns/{actions52}/.style={column name={},column type=C},
    columns/{symba72}/.style={column name={},column type=C},
    columns/{actions72}/.style={column name={},column type=C},
    columns/{symba92}/.style={column name={},column type=C},
    columns/{actions92}/.style={column name={},column type=C},
	%%
    every row 0 column 5/.style={postproc cell content/.style={@cell content={-}}},
    every row 0 column 6/.style={postproc cell content/.style={@cell content={-}}},
    every row 0 column 7/.style={postproc cell content/.style={@cell content={-}}},
    every row 0 column 8/.style={postproc cell content/.style={@cell content={-}}},
    every row 1 column 7/.style={postproc cell content/.style={@cell content={-}}},
    every row 1 column 8/.style={postproc cell content/.style={@cell content={-}}},
]{\datafileANOISE}
}
\resizebox{\textwidth}{!}{\newcolumntype{C}{>{\centering\arraybackslash}p{18mm}}
\pgfplotstabletypeset[
	%every even row/.style={
	%	before row={\rowcolor[gray]{0.9}}
	%},
font=\scriptsize,
	every head row/.style={
    output empty row,
		before row={%
            \midrule
              \multicolumn{1}{c|}{\textbf{\emph{3 const. modified}}}
            & \multicolumn{2}{p{20mm}|}{3 constraints}
			& \multicolumn{2}{p{20mm}|}{5 constraints}
            & \multicolumn{2}{p{20mm}|}{7 constraints}
            & \multicolumn{2}{p{20mm}|}{10 constraints}
            \\\midrule
		},
	},
	%
	columns={length33,symba33,actions33,symba33,actions33,symba33,actions33,symba33,actions33},		
    columns/{length33}/.style={column name={},column type=C},
    columns/{symba33}/.style={column name={},column type=C},
    columns/{actions33}/.style={column name={},column type=C},
    columns/{symba33}/.style={column name={},column type=C},
    columns/{actions33}/.style={column name={},column type=C},
    columns/{symba33}/.style={column name={},column type=C},
    columns/{actions33}/.style={column name={},column type=C},
    columns/{symba33}/.style={column name={},column type=C},
    columns/{actions33}/.style={column name={},column type=C},
	%%
    %every row 0 column 0/.style={postproc cell content/.style={@cell content={1-50}}},
    %every row 1 column 0/.style={postproc cell content/.style={@cell content={51-100}}},
    %every row 2 column 0/.style={postproc cell content/.style={@cell content={101-150}}},
    %every row 3 column 0/.style={postproc cell content/.style={@cell content={151-200}}},
]{\datafileANOISE}
}
\smallskip
\caption{Experimental results. The time (in \emph{ms.}) is the average per trace.}
\label{table:exp_results_synth}
\end{center}
\vspace*{-1.3cm}
\end{table*}


We have developed a planning-based alignment tool that implements the approach discussed in Section~\ref{sec:dccap}.
%
The tool allows us to load existing logs formatted with the XES (eXtensible Event Stream) standard and to import data-aware models previously designed using RuM \cite{AlmanCHMN20}.
%
%A \declare model consists of a set of \emph{constraints}, i.e., \emph{rule templates} applied to activities. Their semantics can be formalized using \LTLf, making them verifiable and executable. \tablename~\ref{table:declarerules} summarizes some \declare templates.
%
%The reader can refer to \cite{declareCSRD09} for a full description of the language.
%
In order to find the minimum cost trace alignment against a pre-specified data-aware Declare model, our tool makes use of the SymBA*-2~\cite{torralba2014symba} planning systems. To produce optimal alignments, SymBA*-2 (winner of the sequential optimizing track at the 2014 Int. Planning Competition) performs a bidirectional A* search.
%
We tested our approach on the grounded version of the problem presented in Section~\ref{ssec:eip}. We performed our experiments with a machine consisting of an Intel Core i7-4770S CPU 3.10GHz Quad Core and 4GB RAM. We used a standard cost function with unit costs for any alignment step that adds/removes activities in/from the input trace or changes an attribute value attached to them, and cost 0 for synchronous moves.

%
To have a sense of the scalability with respect to the ``size'' of the model and the ``noise'' in the traces, we have tested the approach on synthetic logs of different complexity.
%
Specifically, we generated syntectic logs using the log generator presented in \cite{SkydanienkoFGM18}. We defined four Declare models having the same alphabet of activities and containing 3, 5, 7 and 10 data-aware constraints respectively.
%
Then, to create logs containing noise, i.e., behaviors non-compliant with the original Declare models, we changed some of the constraints in these models and generated logs from them. In particular, we modified the original Declare models by replacing 1, 2, and 3 constraints in each model using different strategies. In some cases we replaced a constraint with its negative counterpart (see Table 1); in other cases we replaced a constraint with a weaker constraint; in other cases, we replaced a data condition with its negation. Each modified model was used to generate 5 logs of 1000 traces containing traces of different lengths (i.e., containing 10, 15, 20, 25, and 30 events, respectively).

The results of the experiments can be seen in Table~\ref{table:exp_results_synth}. The alignment time (in ms.) and cost (that corresponds to the amount of \PDDL{ins}/\PDDL{del}/\PDDL{repl} activities in an alignment) refers to the average per trace. The missing values in the table refer to experiments that could not be carried out because
some traces of a certain length (e.g., 10) could not be generated by specific kinds of model (e.g., including 7 or 10 constraints). It is evident from the table that the alignment cost does not affect the performance of the alignment tool as when the noise increases, the execution time does not change. As expected, however, the execution time is slightly sensible to the trace length, and grows exponentially with the number of (data-aware) constraints in the reference model. However, the results suggest that the heuristics adopted by the planner is able to efficiently cope with the above complexity enabling to perform off-line analyses with acceptable performance in case of a reasonably large number of data-aware constraints.

We notice that the declarative models and the event logs generated for the experiments are available for testing and experiments repeatability with our planning-based alignment tool at: \url{https://tinyurl.com/ezd788bb}.

%They include the results obtained by testing the approaches of De Giacomo et al.~\cite{ICAPS2016} and de Leoni et al.~\cite{Leoni2012}.
%
%The results show that both in the real-life and in the synthetic logs the approach of de Leoni et al. is faster for short traces with a small amount of noise.
%
%Conversely, the approach of de Giacomo et al. (which has been tested on the real-life log only) is the slowest one in all tests. Such a poor performance depends on the fact that it needs to determine, for each trace, a bound on the maximum number of instances of each activity needed to align the trace. However, such a bound is not minimal, i.e., more activity instances than those needed for the alignment are incorporated in the planning problem. This dramatically increases the search space.
%When the noise increases and/or the model becomes larger, our planning-based approach outperforms the existing ones by several orders of magnitude. For example, using the synthetic log generated by the Declare model with 20 constraints and 3 constraints modified, containing traces of lengths varying from 151 to 200 events, our approach requires on average around 28.97 seconds (with SymBA*-2) per trace to compute an optimal alignment, while the approach of de Leoni et al. takes 223.47 seconds. This can be explained with the observation that the heuristics adopted by planners are able to efficiently cope with the size of the state space, which is exponential with respect to the size of the model, the amount of noise and the trace length.

%Finally, in order to study the ``boundaries'' of our approach and to understand \emph{how much noise a log needs to contain to make our approach ineffective}, we performed a third assessment by modifying 4 and 6 constraints in each of the Declare models. The results are shown in Table~\ref{table:exp_results_synth} and in the lower plots of Fig.~\ref{fig:time_performance_comparison}. They suggest that the approach is feasible also in case of traces requiring a large number of alignment actions.
%It is interesting to notice that the bidirectional A* search employed in \SYMBA scales better than the blind A* search of \FASTD when the tested models contain a higher number of constraints.

\section{Conclusions}\label{sec:end}
In this paper, we presented an approach tackling conformance checking of log traces over data-aware Declare models. The proposed approach exploits Automated Planning for aligning the log traces and the reference model via a preliminary partitioning of the data space. The experiments show that the performance of the approach is acceptable even when the reference model contains a reasonably large number of data-aware constraints. In addition, since the implemented tool is independent of the planner used to solve the alignment problem, forthcoming improvements in the efficiency of the planners will be automatically transferred to the tool.

Future works will investigate the relationship between planners and approximate path matching techniques \cite{Myers1989} for the implementation of alignment approaches that return not only the optimal alignment but also suboptimal ones that might be of interest for the user. We will also investigate the possibility of performing alignments over data-aware knowledge bases \cite{10.1007/978-3-319-39696-5_18}, which potentially quicken the time required to test the satisfiability of the data conditions by conveniently indexing (i.e., pre-ordering) the payload space \cite{IdreosGNMMK12}. These approaches will still take advantage of the representation of the trace alignment problem as a planning problem presented in this paper. 

%
%
%
%\section{First Section}
%\subsection{A Subsection Sample}
%Please note that the first paragraph of a section or subsection is
%not indented. The first paragraph that follows a table, figure,
%equation etc. does not need an indent, either.
%
%Subsequent paragraphs, however, are indented.
%
%\subsubsection{Sample Heading (Third Level)} Only two levels of
%headings should be numbered. Lower level headings remain unnumbered;
%they are formatted as run-in headings.
%
%\paragraph{Sample Heading (Fourth Level)}
%The contribution should contain no more than four levels of
%headings. Table~\ref{tab1} gives a summary of all heading levels.
%
%\begin{table}
%\caption{Table captions should be placed above the
%tables.}\label{tab1}
%\begin{tabular}{|l|l|l|}
%\hline
%Heading level &  Example & Font size and style\\
%\hline
%Title (centered) &  {\Large\bfseries Lecture Notes} & 14 point, bold\\
%1st-level heading &  {\large\bfseries 1 Introduction} & 12 point, bold\\
%2nd-level heading & {\bfseries 2.1 Printing Area} & 10 point, bold\\
%3rd-level heading & {\bfseries Run-in Heading in Bold.} Text follows & 10 point, bold\\
%4th-level heading & {\itshape Lowest Level Heading.} Text follows & 10 point, italic\\
%\hline
%\end{tabular}
%\end{table}
%
%
%\noindent Displayed equations are centered and set on a separate
%line.
%\begin{equation}
%x + y = z
%\end{equation}
%Please try to avoid rasterized images for line-art diagrams and
%schemas. Whenever possible, use vector graphics instead (see
%Fig.~\ref{fig1}).
%
%\begin{figure}
%\includegraphics[width=\textwidth]{fig1.eps}
%\caption{A figure caption is always placed below the illustration.
%Please note that short captions are centered, while long ones are
%justified by the macro package automatically.} \label{fig1}
%\end{figure}
%
%\begin{theorem}
%This is a sample theorem. The run-in heading is set in bold, while
%the following text appears in italics. Definitions, lemmas,
%propositions, and corollaries are styled the same way.
%\end{theorem}
%%
%% the environments 'definition', 'lemma', 'proposition', 'corollary',
%% 'remark', and 'example' are defined in the LLNCS documentclass as well.
%%
%\begin{proof}
%Proofs, examples, and remarks have the initial word in italics,
%while the following text appears in normal font.
%\end{proof}
%For citations of references, we prefer the use of square brackets
%and consecutive numbers. Citations using labels or the author/year
%convention are also acceptable. The following bibliography provides
%a sample reference list with entries for journal
%articles~\cite{ref_article1}, an LNCS chapter~\cite{ref_lncs1}, a
%book~\cite{ref_book1}, proceedings without editors~\cite{ref_proc1},
%and a homepage~\cite{ref_url1}. Multiple citations are grouped
%\cite{ref_article1,ref_lncs1,ref_book1},
%\cite{ref_article1,ref_book1,ref_proc1,ref_url1}.
%
% ---- Bibliography ----
%
% BibTeX users should specify bibliography style 'splncs04'.
% References will then be sorted and formatted in the correct style.
%
\bibliographystyle{splncs04}
 \bibliography{biblio}
%

\end{document}
